%!TEX root = Progetto.tex
\section{Progettazione Logica}
	\subsection{Tavola dei Volumi}
		\subsubsection{Tavola dei Volumi}

			I volumi delle entità e delle relazioni sono stati stimati facendo riferimento ad un ciclo di vita della base di dati di circa 3 anni. In tale calcolo abbiamo considerato che alcune entità sono soggette ad una iniziale migrazione di dati (come ad esempio l'entità \emph{Fornitore} o \emph{Componente}).
		
			\begin{longtable}{| p{4cm} | p{4cm} | p{4cm} |}


				\hline
				\textbf{Concetto} & 
				\textbf{Tipo} & 
				\textbf{Volume} \\
				\hline
				
				\endfirsthead
				
				\hline
				\textbf{Concetto} & 
				\textbf{Tipo} & 
				\textbf{Volume} \\
				\hline
				\endhead
				
				Persona       & E & 495  \\ \hline
				Cliente       & E & 450  \\ \hline
				Privato       & E & 300  \\ \hline
				Azienda       & E & 150  \\ \hline
				Fornitore     & E & 42   \\ \hline
				Operatore     & E & 3    \\ \hline
				Autovettura   & E & 585  \\ \hline
				Preventivo    & E & 1500 \\ \hline
				Componente    & E & 472  \\ \hline
				Fornitura     & E & 1500 \\ \hline
				Ordine        & E & 300  \\ \hline
				Magazzino     & E & 189  \\ \hline
				Prestazione   & E & 1425 \\ \hline
				Turno         & E & 3000 \\ \hline
				Transazione   & E & 3747 \\ \hline
				Fattura       & E & 1425 \\ \hline
				Recapito      & E & 1077 \\ \hline
				Esecuzione    & R & 1425 \\ \hline
				Previsione    & R & 9000 \\ \hline
				Riferimento   & R & 1500 \\ \hline
				Formazione    & R & 1500 \\ \hline
				Composizione  & R & 472  \\ \hline
				Utilizzo      & R & 8550 \\ \hline
				Occupazione   & R & 2850 \\ \hline
				Presenza      & R & 3000 \\ \hline
				Possesso      & R & 585  \\ \hline
				Richiesta     & R & 1500 \\ \hline
				Acquisto      & R & 300  \\ \hline
				Rubrica       & R & 1077 \\ \hline
				Acconto       & R & 750  \\ \hline
				Registrazione & R & 1425 \\ \hline
				Pagamento     & R & 1425 \\ \hline
				Versamento    & R & 300  \\ \hline
				Salario       & R & 72   \\ \hline
				
			\end{longtable}
			
		\subsubsection{Tavola delle Operazioni}
		
			\begin{longtable}{| p{6.2cm} | p{6.2cm} |}

				\hline
				\textbf{Operazione} & 
				\textbf{Frequenza} \\
				\hline
				
				\endfirsthead
				
				\hline
				\textbf{Operazione} & 
				\textbf{Frequenza} \\
				\hline
				
				\endhead
				
				% Inserimenti
				\ref{op:new_cliente} & 3 volte a settimana \\ \hline
				\ref{op:new_privato} & 2 volte a settimana \\ \hline
				\ref{op:new_azienda} & 1 volta a settimana \\ \hline
				\ref{op:new_auto} & 3 volte a settimana \\ \hline
				\ref{op:new_fornitore} & 4 volte l'anno \\ \hline
				\ref{op:new_componente} & 2 volte al mese \\ \hline
				\ref{op:new_ordine} & 2 volte a settimana \\ \hline
				\ref{op:new_fornitura} & 10 volte a settimana \\ \hline
				\ref{op:new_preventivo} & 10 volte a settimana \\ \hline
				\ref{op:new_prestazione} & 10 volte a settimana \\ \hline
				\ref{op:new_fattura} & 10 volte a settimana \\ \hline
				\ref{op:new_operatore} & 1 volta all'anno \\ \hline
				\ref{op:new_recapito} & 80 volte ogni 3 mesi \\ \hline
				\ref{op:new_presenza} & 2 volte al giorno per ogni operatore \\ \hline
				\ref{op:new_transazione} & 18 volte a settimana \\ \hline

				% Assegnazioni
				\ref{op:ass_componente_preventivo} & 60 volte a settimana \\ \hline
				\ref{op:ass_componente_prestazione} & 60 volte a settimana \\ \hline
				\ref{op:ass_fornitura_ordine} & 10 volte a settimana \\ \hline

				% Aggiornamenti
				\ref{op:edit_cliente} & 30 volte l'anno \\ \hline
				\ref{op:edit_fornitore} & 2 volte l'anno \\ \hline
				\ref{op:edit_operatore} & 1 volta l'anno \\ \hline
				\ref{op:edit_componente} & 5 volte al mese \\ \hline

				% Consultazioni
				\ref{op:show_collaudo} & 1 volta a settimana \\ \hline
				\ref{op:show_revisione} & 1 volta a settimana \\ \hline
				\ref{op:show_transazioni} & 1 volta a settimana \\ \hline
				\ref{op:show_riparazioni} & 2 volte al giorno \\ \hline
				\ref{op:show_preventivi} & 2 volte al giorno \\ \hline
				\ref{op:check_componente} & 4 volte al giorno \\ \hline
				\ref{op:check_turni} & 1 volta al mese \\ \hline
				\ref{op:list_componenti} & 1 volta a settimana \\ \hline
				\ref{op:stats_componenti} & 2 volte al mese \\ \hline
				\ref{op:tobuy_componenti} & 1 volta a settimana \\ \hline
				\ref{op:show_recapiti_cliente} & 10 volte a settimana \\ \hline
				\ref{op:show_recapiti_fornitore} & 2 volte a settimana \\ \hline
				\ref{op:show_recapiti_operatore} & 1 volta a settimana \\ \hline
				\ref{op:list_fatture_pending} & 1 volta al giorno \\ \hline
				\ref{op:todo_list} & 1 volta al giorno \\ \hline
				\ref{op:show_fattura} & 10 volte a settimana \\ \hline
				\ref{op:calc_stipendio_operatore} & 1 volta al mese per ogni operatore \\\hline
				\ref{op:stats_prevetivi_prestazioni} & 1 volta a settimana \\ \hline
				\ref{op:stats_costi} & 1 volta a settimana \\ \hline

			\end{longtable}
			
	\subsection{Ristrutturazione dello Schema Concettuale}
		\subsubsection{Analisi delle Derivazioni e della Ridondanza}
			Ridondanze da analizzare e decidere se introdurre:
			\begin{enumerate}
				\item Totale in fattura
				\item Paga in operatore
				\item Totale negli ordini
			\end{enumerate}
			
		\subsubsection{Eliminazione delle Generalizzazioni}
			
			Da eliminare la generalizzazione Persona. Scorpararla in Clienti, Fornitori, Operatori. Accorpare Privati e Aziende in Cliente.
			
		\subsubsection{Partizionamento e Accorpamento di Concetti}
	\subsection{Scelta degli Identificatori Principali}
		
		Tabella con gli identificatori
		
	\subsection{Normalizzazione}
		Da mettere prima o dopo la traduzione verso il modello relazionale
	\subsection{Traduzione verso il Modello Relazionale}
