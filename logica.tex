\section{Progettazione Logica}
	\subsection{Tavola dei Volumi}
		\subsubsection{Tavola dei Volumi}
		
			\begin{longtable}{| p{4cm} | p{4cm} | p{4cm} |}
				
				\hline
				\textbf{Concetto} & 
				\textbf{Tipo} & 
				\textbf{Volume} \\
				\hline
				
				\endfirsthead
				
				\hline
				\textbf{Concetto} & 
				\textbf{Tipo} & 
				\textbf{Volume} \\
				\hline
				
				\endhead
			
				Persona			& E & 161\\ \hline
				Cliente 		& E & 156\\ \hline
				Fornitore 		& E & 4\\ \hline
				Operatore 		& E & 1\\ \hline
				Autovettura 	& E & 208\\ \hline
				Preventivo		& E & 520\\ \hline
				Componente		& E & 424\\ \hline
				Fornitura 		& E & 260\\ \hline
				Ordine			& E & 52 \\ \hline
				Magazzino 		& E & 636\\ \hline
				Prestazione 	& E & 494\\ \hline
				Turno 			& E & 365\\ \hline
				Transazione 	& E & 1026 \\ \hline
				Fattura 		& E & 494\\ \hline
				Recapito		& E & 165\\ \hline
				Esecuzione		& R & 494\\ \hline
				Previsione		& R & 3120 \\ \hline
				Riferimento 	& R & 260\\ \hline
				Formazione		& R & 260\\ \hline
				Composizione	& R & 233\\ \hline
				Utilizzo		& R & 2964 \\ \hline
				Occupazione 	& R & 988\\ \hline
				Presenza		& R & 365\\ \hline
				Possesso		& R & 208\\ \hline
				Richiesta 		& R & 520\\ \hline
				Acquisto		& R & 52 \\ \hline
				Rubrica			& R & 165\\ \hline
				Acconto 		& R & 260\\ \hline
				Registrazione 	& R & 494\\ \hline
				Pagamento 		& R & 494\\ \hline
				Versamento		& R & 52 \\ \hline
				Salario 		& R & 1\\ \hline
				
			\end{longtable}
			
		\subsubsection{Tavola delle Operazioni}
		
			\begin{longtable}{| p{6cm} | p{6cm} |}
				
				\hline
				\textbf{Operazione} & 
				\textbf{Frequenza} \\
				\hline
				
				\endfirsthead
				
				\hline
				\textbf{Operazione} & 
				\textbf{Frequenza} \\
				\hline
				
				\endhead
				
				1 & 6 volte a settimana \\ \hline
				2 & 5 volte a settimana \\ \hline
				3 & 2 volte all'anno\\ \hline
				4 & 2 volte al mese \\ \hline
				5 & 1 volta a settimana \\ \hline
				6 & 5 volte a settimana \\ \hline
				7 & 10 volte a settimana\\ \hline
				8 & 3 volte a settimana \\ \hline
				9 & 9 volte a settimana \\ \hline
				10 & 10 volte a settimana\\ \hline
				11 & 1 volta all'anno\\ \hline
				12 & 1 volta per ogni nuovo cliente e per ogni operatore, in media 2 volte per ogni fornitore\\ \hline
				13 & 1 volta al giorno per operatore \\ \hline
				14 & 1 volta ogni 2 preventivi, 1 volta al mese per ogni operatore, 1 volta per ogni fornitura, 1 una volta per ogni prestazione \\ \hline
				15 & 6 volte per ogni preventivo \\ \hline
				16 & 6 volte per ogni prestazione\\ \hline
				17 & 2 volte all'anno\\ \hline
				18 & 1 volta all'anno\\ \hline
				19 & 1 volta all'anno\\ \hline
				20 & 1 volta al mese \\ \hline
				21 & 1 volta a settimana \\ \hline
				22 & 1 volta a settimana \\ \hline
				23 & 1 volta a settimana \\ \hline
				24 & 2 volte al giorno \\ \hline
				25 & 2 volte al giorno \\ \hline
				26 & 4 volte al giorno \\ \hline
				27 & 1 volta al mese \\ \hline
				28 & 1 volta al mese \\ \hline
				29 & 2 volte al mese \\ \hline
				30 & 1 volta a settimana \\ \hline
				31 & 1 volta al giorno \\ \hline
				32 & 1 volta a settimana \\ \hline
				33 & 1 volta a settimana \\ \hline
				34 & 1 volta al giorno \\ \hline
				35 & 1 volta al giorno \\ \hline
				36 & 10 volte a settimana\\ \hline
				37 & 1 volta al mese \\ \hline
				38 & 1 volta a settimana \\ \hline
				39 & 1 volta a settimana \\ \hline
		
			\end{longtable}
		
			
	\subsection{Ristrutturazione dello Schema Concettuale}
		\subsubsection{Analisi delle Derivazioni e della Ridondanza}
			Ridondanze da analizzare e decidere se introdurre:
			\begin{enumerate}
				\item Totale in fattura
				\item Paga in operatore
				\item Totale negli ordini
			\end{enumerate}
			
		\subsubsection{Eliminazione delle Generalizzazioni}
			
			Da eliminare la generalizzazione Persona. Scorpararla in Clienti, Fornitori, Operatori. Accorpare Privati e Aziende in Cliente.
			
		\subsubsection{Partizionamento e Accorpamento di Concetti}
	\subsection{Scelta degli Identificatori Principali}
		
		Tabella con gli identificatori
		
	\subsection{Normalizzazione}
		Da mettere prima o dopo la traduzione verso il modello relazionale
	\subsection{Traduzione verso il Modello Relazionale}
