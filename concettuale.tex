\section{Progettazione Concettuale}
	
	\subsection{Strategia di Progetto}
		\emph{Inside-Out} e \emph{Top-Down} per lo scheletro.
		\emph{Inside-Out} per i raffinamenti successivi.
						
		{\color{red}{Questa sezione va ampliata e corretta}}
		
	\subsection{Individuazione dello Scheletro dello Schema ER}
		
		Dalle specifiche che abbiamo formulato risulta che uno dei punti fondamentali da affrontare è quello della memorizzazione dei preventivi emessi dall'attività.
		Ad ogni \emph{Prestazione} effettuata, corrisponde un \emph{Preventivo} precedentemente emesso.
		
		\begin{figure}[H]
			\centering
			\includegraphics[width=9cm]{images/diagrams/preventivo_prestazione.png}
		\end{figure}
		
		Alla formulazione di ogni preventivo, si fa una stima dei componenti che si reputa saranno necessari per eseguire la prestazione preventivata. Non sempre tale previsione è completamente completamente esatta, generalmente i componenti effettivamente utilizzati in una prestazione sono diversi da quelli previsti in un preventivo.
		L'associazione tra i \emph{Componenti} e il \emph{Preventivo} risulta immediata.
		
		\begin{figure}[H]
			\centering
			\includegraphics[width=9cm]{images/diagrams/preventivo_componente.png}
		\end{figure}
		
		Prima di esplicitare l'associazione tra le prestazioni eseguite ed i componenti effettivamente utilizzati, affrontiamo la questione degli ordini e del magazzino.
		
		L'acquisto di articoli presso un fornitore è formalizzato in un ordine, il quale, come da specifiche, è organizzato in più forniture, ovvero insiemi di articoli di uno stesso componente.
		Ad un \emph{Ordine} sono associate una o più \emph{Forniture} di articoli, ognuna delle quali fanno riferimento ad un \emph{Componente}.
		
		\begin{figure}[H]
			\centering
			\includegraphics[width=11.5cm]{images/diagrams/componente_ordine.png}
		\end{figure}
		
		Il magazzino, nella realtà, è composto dai vari articoli acquistati che sono in attesa di essere utilizzati. La classificazione degli articoli avviene, in primo luogo per componente, in secondo luogo per fornitura d'appartenenza. Non vi è così il bisogno di registrare ogni articolo individualmente, ma basterà riferirsi alle relative forniture.
		Il \emph{Magazzino} è una composizione di \emph{Forniture} i cui articoli sono depositati in attesa di essere utilizzati.
		
		\begin{figure}[H]
			\centering
			\includegraphics[width=9cm]{images/diagrams/magazzino_fornitura.png}
		\end{figure}
		
		Per identificare con precisione quali articoli sono stati utilizzati per l'esecuzione di una prestazione, sarà sufficiente riferirsi alla fornitura relativa agli stessi. Da questa si ottengono le informazioni sul componente (quindi il prezzo di vendita e il tempo di validità) e sulla data d'acquisto.
		Ad ogni utilizzo, si provvederà ad aggiornare le quantità rimanenti degli articoli delle forniture utilizzate.
		Per l'esecuzione di una \emph{Prestazione} si possono utilizzare gli articoli di più \emph{Forniture}.
		
		\begin{figure}[H]
			\centering
			\includegraphics[width=9cm]{images/diagrams/prestazione_fornitura.png}
		\end{figure}
		
		Passiamo alla questione degli operatori. Una prestazione viene eseguita da uno o più operatori. Di ogni operatore si vuole tener traccia dei turni di lavoro effettuati.
		Alla \emph{Prestazione}, saranno associati uno o più \emph{Operatori} ad ognuno dei quali sono associati i relativi \emph{Turni} di lavoro.
		
		\begin{figure}[H]
			\centering
			\includegraphics[width=9cm]{images/diagrams/operatore_turno_prestazione.png}
		\end{figure}
		
		Occupiamoci ora delle zone periferiche dello schema. Un preventivo viene effettuato quando un cliente richiede un intervento alla propria auto.
		Ad ogni \emph{Cliente} vengono associate una o più \emph{Autovetture}. Ogni \emph{Preventivo} si riferisce ad una specifica \emph{Autovettura}.
		
		\begin{figure}[H]
			\centering
			\includegraphics[width=9cm]{images/diagrams/cliente_autovettura_preventivo.png}
		\end{figure}
		
		Ogni \emph{Ordine} viene effettuato presso un \emph{Fornitore}.
		
		\begin{figure}[H]
			\centering
			\includegraphics[width=9cm]{images/diagrams/ordine_fornitore.png}
		\end{figure}
		
		Notiamo che \emph{Cliente} rappresenta sia \emph{Privati} che \emph{Aziende} (rispettivamente, clienti non dotati di partita iva e clienti dotati di partita iva).
		
		\begin{figure}[H]
			\centering
			\includegraphics[width=7.5cm]{images/diagrams/cliente.png}
		\end{figure}
		
		Inoltre \emph{Clienti}, \emph{Fornitori} ed \emph{Operatori} possono essere generalizzati dall'entità \emph{Persona}.
		
		\begin{figure}[H]
			\centering
			\includegraphics[width=10cm]{images/diagrams/persona.png}
		\end{figure}
		
		Ad ogni \emph{Persona} saranno associati uno o più \emph{Recapiti}.
		
		\begin{figure}[H]
			\centering
			\includegraphics[width=9cm]{images/diagrams/persona_recapito.png}
		\end{figure}
		
		Possiamo concludere lo sviluppo della struttura del diagramma ER affrontando la questione delle transazioni. Avviene una \emph{Transazione} ogni volta che viene versato un acconto per un \emph{Preventivo}, ogni volta che viene saldata la \emph{Fattura} di una \emph{Prestazione}, ogni volta che viene pagato un \emph{Ordine} ed ogni volta che viene pagato il salario di un \emph{Operatore}.
		
		\begin{figure}[H]
			\centering
			\includegraphics[width=12cm]{images/diagrams/transazione.png}
		\end{figure}
			
		Il diagramma in figura \ref{fig:scheletro_er} rappresenta lo scheletro dello schema ER.
		
		\begin{sidewaysfigure}
			\centering
			\includegraphics[width=22cm]{images/diagrams/schema.png}
			\caption{Scheletro del diagramma ER}
			\label{fig:scheletro_er}
		\end{sidewaysfigure}
	
	\subsection{Sviluppo delle Componenti dello Schema}
	
		Ottenuto lo scheletro generale del diagramma ER procediamo ad esplicitare, delle entità principali, l'insieme degli attributi che ognuna di esse possiede.
		Una volta sviluppati gli attributi delle entità principali svilupperemo le relationship che le legano, raggruppandole tra loro sulla falsariga dei modelli elaborati allo step precedente.
		
		Gli attributi di ogni entità derivano ovviamente dall'analisi dei requisi della base di dati, quindi le successive sezioni si limitano a presentare graficamente i successivi raffinamenti sulle componenti dello schema ER, con l'aggiunta di eventuali descrizioni per integrare ciò che non era stato già detto.
		Per una lettura più accurata e rigorosa di tali diagrammi si rimanda alla consultazione della documentazione di supporto (Dizionario dei Dati alla sezione \ref{sec:data_dict} e Regole Aziendali \ref{sec:business_rules}).
		
		\begin{description}
			\item[NB]
				Ogni generalizzazione effettuata è da considerarsi totale.
		\end{description}
		
		\subsubsection{Persona}
		
			In figura \ref{fig:persona} troviamo lo sviluppo degli attributi dell'entità \emph{Persona} e delle relative entità che la estendono.
						
			\begin{figure}[H]
				\centering
				\includegraphics[width=12cm]{images/finitures/persona.png}
				\caption{Sviluppo di Persona}
				\label{fig:persona}
			\end{figure}
			
			Di ogni \emph{Persona} coinvolta nelle attività dell'azienda, suddivisibili in \emph{Clienti}, \emph{Fornitori} ed \emph{Operatori}, è necessario memorizzare all'interno della base di dati il codice fiscale e l'indirizzo di riferimento.
		
		\subsubsection{Autovettura}
			
			Continuiamo con gli attributi che caratterizzano l'entità \emph{Autovettura} (Diagramma in figura \ref{fig:autovettura})
			
			\begin{figure}[H]
				\centering
				\includegraphics[width=9cm]{images/finitures/autovettura.png}
				\caption{Sviluppo dell'entità Autovettura}
				\label{fig:autovettura}
			\end{figure}
			
			Si è scelto l'attrubuto \emph{Targa} come chiave primaria della relazione piuttosto che l'attributo \emph{Telaio} (numero di serie del telaio) nonostante anche quest'ultimo identifichi univocamente l'autovettura. Riteniamo che sia più agevole identificare un'autovettura attraverso la targa poichè tale informazione è più facilmente reperibile rispetto al seriale del telaio.
			
		\subsubsection{Preventivo}
		
			In figura \ref{fig:preventivo}, il diagramma espone gli attributi dell'entità \emph{Preventivo}.
			
			\begin{figure}[H]
				\centering
				\includegraphics[width=9cm]{images/finitures/preventivo.png}
				\caption{Sviluppo dell'entità Preventivo}
				\label{fig:preventivo}
			\end{figure}
			
		\subsubsection{Prestazione}
			
			Gli attributi dell'entità \emph{Prestazione} vengono esplicitati dal diagramma in figura \ref{fig:prestazione}.
		
			\begin{figure}[H]
				\centering
				\includegraphics[width=9cm]{images/finitures/prestazione.png}
				\caption{Sviluppo dell'entità Prestazione}
				\label{fig:prestazione}
			\end{figure}
			
			Dovendo tener traccia del preventivo di riferimento a fronte di una prestazione fornita, abbiamo scelto l'attributo \emph{Preventivo} (il codice identificativo del preventivo) come chiave primaria, dal momento che non vi possono essere più prestazioni a fronte dello stesso preventivo.
		
		\subsubsection{Componente}
			
			Nel diagramma in figura \ref{fig:componente} troviamo l'entità \emph{Componente} ed i relativi attributi.
			
			\begin{figure}[H]
				\centering
				\includegraphics[width=9cm]{images/finitures/componente.png}
				\caption{Sviluppo di Componente}
				\label{fig:componente}
			\end{figure}
		
		\subsubsection{Fattura}
		
			Nel diagramma in figura \ref{fig:fattura}, l'entità \emph{Fattura} ed i suoi attributi.
		
			\begin{figure}[H]
				\centering
				\includegraphics[width=7cm]{images/finitures/fattura.png}
				\caption{Sviluppo di Transazione}
				\label{fig:fattura}
			\end{figure}
		
		\subsubsection{Transazione}
			
			Il diagramma in figura \ref{fig:transazione} raffigura lo sviluppo degli attributi dell'entità \emph{Transazione}.
						
			\begin{figure}[H]
				\centering
				\includegraphics[width=4.5cm]{images/finitures/transazione.png}
				\caption{Sviluppo di Transazione}
				\label{fig:transazione}
			\end{figure}
		
		\subsubsection{Raffinamenti Successivi}
			
			Esplicitati gli attributi delle principali entità, procediamo a legarle tra loro sviluppando le relationship ed alcune entità minori.
			
			Ripercorrendo i passi dello sviluppo dello scheletro del diagramma ER, partiamo dalle relationship che legano le entità \emph{Cliente}, \emph{Autovettura}, \emph{Preventivo} (diagramma in figura \ref{fig:cliente_autovettura_preventivo}).		
		
			\begin{figure}[H]
				\centering
				\includegraphics[width=13cm]{images/finitures/cliente_autovettura_preventivo.png}
				\caption{Sviluppo delle relationship che legano Cliente, Autovettura e Preventivo}
				\label{fig:cliente_autovettura_preventivo}
			\end{figure}
			
			Chiaramente ad ogni cliente registrato, saranno associate una o più autovetture di sua propiertà. Ad ogni autovettura saranno associati uno o più preventivi di interventi riferiti all'autovettura stessa (Diagramma in figura \ref{fig:cliente_autovettura_preventivo}).
			
			Se l'intervento preventivato viene realizzato, al preventivo sarà associata una ed una sola prestazione. La stipulazione del preventivo non è vincolante nei confronti del cliente, quindi non è vero che ad ogni preventivo corrisponde una prestazione (Diagramma in figura \ref{fig:preventivo_prestazione}).
			
			\begin{figure}[H]
				\centering
				\includegraphics[width=13cm]{images/finitures/preventivo_prestazione.png}
				\caption{Sviluppo della relationship che lega Preventivo e Prestazione}
				\label{fig:preventivo_prestazione}
			\end{figure}
			
			I componenti previsti nelle riparazioni vengono descritti tramite la relazione \emph{Previsione} che lega le entità \emph{Componente} e \emph{Preventivo}. In ogni preventivo si può prevedere di utilizzare nessuno, uno o più componenti. L'utilizzo di uno stesso componente può essere previsto - ovviamente - nella formulazione di più preventivi.
			Si consulti il diagramma in figura \ref{fig:preventivo_componente}.
			
			\begin{figure}[H]
				\centering
				\includegraphics[width=13cm]{images/finitures/preventivo_componente.png}
				\caption{Sviluppo di Preventivo e Componente}
				\label{fig:preventivo_componente}
			\end{figure}
			
			L'attributo "Ubicazione" della relazione \emph{Previsione} rappresenta l'ubicazione dei componenti utilizzati nelle installazioni di nuovi impianti (si consultino anche le specifiche riguardanti i preventivi alla sottosezione "Frasi relative ai Preventivi" \ref{sec:frasi_preventivi}).
			
			Per la registrazione degli articoli acquistati sono state introdotte in fase di sviluppo dello scheletro dello schema ER le entità \emph{Ordine} e \emph{Fornitura}. Tali entità, prese singolarmente, sono poco significative, essendo fortemente legate tra di loro (si faccia riferimento al diagramma in figura \ref{fig:fornitore_ordine_fornitura_componente}).
			
			I contratti di acquisto con i fornitori vengono modellati dall'entità \emph{Ordine}. Naturalmente presso lo stesso fornitore si possono effettuare più ordini, ma un ordine si riferisce ad un singolo fornitore. \emph{Ordine} e \emph{Fornitore} sono legati dalla relationship \emph{Acquisto}.
			
			\begin{figure}[H]
				\centering
				\includegraphics[width=13cm]{images/finitures/fornitore_ordine_fornitura_componente}
				\caption{Sviluppo delle relazioni che legano il Fornitore, l'Ordine d'acquisto, le Forniture e i Componenti}
				\label{fig:fornitore_ordine_fornitura_componente}
			\end{figure}
			
			Ogni ordine è composto da una o più forniture, le quali, a loro volta, sono composte da uno o più articoli dello stesso componente. Ad ogni istanza dell'entità \emph{Fornitura} si associa - tramite la relationship \emph{Riferimento} - una ed una sola istanza dell'entità \emph{Componente}. Di contro, lo stesso componente può essere acquistato in diverse forniture.
			Quindi ogni istanza di \emph{Fornitura} sarà associata ad una ed una sola istanza di \emph{Ordine} tramite la relationship \emph{Formazione}. Un ordine vedrà associate a sè una o più forniture.
			
			\begin{figure}[H]
				\centering
				\includegraphics[width=13cm]{images/finitures/componente_fornitura_ordine_magazzino.png}
				\caption{Introduzione del Magazzino}
				\label{fig:componente_fornitura_ordine_magazzino}
			\end{figure}
			
			Un'ulteriore entità da aggiungere a questo gruppo è quella del magazzino. Abbiamo definito \emph{Magazzino} come una raccolta di forniture attive, cioè di forniture i quali articoli sono ancora presenti nel magazzino fisico, pronti per essere utilizzati. Legando \emph{Magazzino} con \emph{Fornitura} si modella tale associazione. La relationship \emph{Composizione} associa ad ogni istanza di \emph{Magazzino} una ed una sola istanza di \emph{Fornitura}. Da notare che il codice della fornitura e il codice del componente di riferimento, formano la chiave primaria di tale entità.
			Fare riferimento al diagramma in figura \ref{fig:componente_fornitura_ordine_magazzino}.
			
			Da notare che l'attributo \emph{quantità} dell'entità \emph{Magazzino} descrive il numero di articoli di uno specifico componente, acquistati in una certa fornitura, ancora disponibili in magazzino.
			
			All'esecuzione di una prestazione, come da specifiche, è necessario specificare il tipo e la quantità di articoli utilizzati. La prima soluzione che ci è sembrata valida è stata quella di associare, ad ogni istanza dell'entità \emph{Prestazione} le istanze interessate dell'entità \emph{Componente} attraverso la relationship \emph{Utilizzo}. Tale relationship avrebbe avuto l'attributo "Quantità", necessario per specificare la quantità degli articoli utilizzati per ogni componente.
			
			Tuttavia, tale design si è rivelato non adeguato a soddisfare tutte le specifiche. Il problema più evidente risiedeva nel fatto che, essendo le istanze dell'entità \emph{Componente} composte da informazioni descrittive, pressocchè invarianti (eccezion fatta per quanto riguarda il "Prezzo di Vendita"), non vi è il modo per risalire al preciso articolo fisico utilizzato nella riparazione\footnote{Non potendo risalire al preciso articolo utilizzato, non si ha a disposizione la data di acquisto, quindi viene meno la realizzabilità del meccanismo che permette di utilizzare per primi gli articoli dei componenti la cui data di scadenza è più vicina di altri.}.
			
			All'entità \emph{Prestazione} vengono quindi associate zero, una o più istanze dell'entità \emph{Fornitura}, avendo così a disposizione sia le informazioni che descrivono genericamente il componente, sia quelle che caratterizzano con precisione l'articolo utilizzato nella prestazione. 

			\begin{figure}[H]
				\centering
				\includegraphics[width=11.5cm]{images/finitures/prestazione_fornitura.png}
				\caption{Utilizzo di componenti un una prestazione}
				\label{fig:ordine_fornitore}
			\end{figure}
			
			L'attributo "Quantità" della relationship \emph{Utilizzo} non necessita di ulteriori spiegazioni, mentre l'attributo "Prezzo Unitario" si rende necessario, in quanto il prezzo di vendita di un componente, ragionevolmente, varia nel tempo.
			
			Ad esecuzione ultimata di una prestazione avviene la registrazione della fattura. Ad ogni istanza dell'entità \emph{Prestazione} sarà associata, tramite la relationship \emph{Registrazione} obbligatoriamente una ed una sola istanza dell'entità \emph{Fattura}. 		
			Le istanze dell'entità \emph{Fattura} vengono identificate dalla coppia di attributi "Numero Progressivo" ed "Anno", così come avviene nella realtà di interesse\footnote{Le fatture vengono identificate dall'anno di emissione e dal numero progressivo. Ogni anno tale numero viene azzerato.}.
			
			Sul significato di altri attributi ("Tipo Pagamento", "Sistema Pagamento", "Stato Pagamento", "Data Scadenza") presenti nel diagramma in figura \ref{fig:prestazione_fattura}, si rimanda alla lettura dell'apposita documentazione nelle sezioni successive.
			
			\begin{figure}[H]
				\centering
				\includegraphics[width=11.5cm]{images/finitures/prestazione_fattura.png}
				\caption{Sviluppo della relationship tra Prestazione e Fattura}
				\label{fig:prestazione_fattura}
			\end{figure}
			
			Gli sviluppi dei diagrammi introdotti fin'ora permettono di affrontare il legame di \emph{Transazione} con le altre entità. A quest'ultima si possono associare istanze di tutte le entità che modellano dati di porzioni di processo che prevedono il verificarsi di transazioni monetarie. Alla stipulazione di un preventivo può essere richiesto il versamento di un acconto, alla consegna gli ordine sarà necessario effettuare una versamento al fornitore, mensilmente bisognerà registrare gli stipendi versati agli operatori e quando una fattura viene saldata bisognerà registrare tale transazione di denaro.
			
			\begin{figure}[H]
				\centering
				\includegraphics[width=13cm]{images/finitures/transazione_fattura_preventivo_operatore_ordine.png}
				\caption{Sviluppo delle relationship con cui Transazione si lega alle altre entità}
				\label{fig:transazione_fattura_preventivo_operatore_ordine}
			\end{figure}
			
			Nel diagramma in figura \ref{fig:transazione_fattura_preventivo_operatore_ordine} vi è la rappresentazione di come le entità \emph{Preventivo}, \emph{Ordine}, \emph{Operatore}, \emph{Fattura} vengono associate a \emph{Transazione}.
			
			Esaminiamo le ultime componenti del diagramma ER che non sono state ancora analizzate.
			
			In figura \ref{fig:operatore_prestazione} il diagramma descrive la relationship \emph{Occupazione} che associa le istanze di \emph{Prestazione} a quelle di \emph{Operatore}. Ad ogni istanza di \emph{Prestazione} infatti devono essere associate una o più instanze di \emph{Operatore}, in modo da tener traccia dei dipendenti che sono stati impiegati nell'esecuzione della prestazione ad un'autovettura. Ovviamente la stessa istanza di \emph{Operatore} può essere associata a più istanze di \emph{Prestazione}.
			
			\begin{figure}[H]
				\centering
				\includegraphics[width=11.5cm]{images/finitures/operatore_prestazione.png}
				\caption{Sviluppo della relationship tra Prestazione e Operatore}
				\label{fig:operatore_prestazione}
			\end{figure}
			
			Riguardo gli operatori è necessario, come da specifiche, registrarne le presenze e gli orari di lavoro. L'entità \emph{Turno}, legata ad \emph{Operatore} tramite la relationship \emph{Presenza} (diagramma in figura \ref{fig:operatore_turno}), assolve tale funzione.
			
			\begin{figure}[H]
				\centering
				\includegraphics[width=11.5cm]{images/finitures/operatore_turno.png}
				\caption{Turni degli Operatori}
				\label{fig:operatore_turno}
			\end{figure}
			
			L'ultimo punto da sviluppare consiste nella gestione dei recapiti, di qualunque natura essi siano. L'entità \emph{Recapito} è costituita dagli attributi "Recapito" (che è anche chiave primaria) e dall'attributo "Tipo" (consultare le Regole Aziendali alla sezione \ref{sec:business_rules} per i valori che tale attributo può assumere).
			Ad ogni istanza di \emph{Persona} devono essere associati una o più istanze di \emph{Recapito}.
			
			\begin{figure}[H]
				\centering
				\includegraphics[width=11.5cm]{images/finitures/persona_rubrica.png}
				\caption{Turni degli Operatori}
				\label{fig:persona_recapito}
			\end{figure}
	
	% Diagramma definitivo
	\subsection{Diagramma Entity-Relationship}
		
		{\color{red}{Inserire il diagramma finale}}

	\subsection{Analisi Qualitativa dello Schema ER}
	
	% Sezione documentativa della progettazione concettuale
	\subsection{Dizionario dei Dati}
	\label{sec:data_dict}
		
		\subsubsection{Entità}
		\label{sec:entities}
		
			\begin{longtable}{| p{2.5cm} | p{4.5cm} | p{2cm} | p{2.5cm} |}
				\hline
				
					\textbf{Nome} & \textbf{Descrizione} & \textbf{Attributi} & \textbf{Identificatore} \\ \hline
					
					Nome entità & 
					Descrizione entità & 
					Attributi entità & 
					Identificatore entità
					\\
					
				\hline
			\end{longtable}

		\subsubsection{Relazioni}
		\label{sec:relationships}
	
	\subsection{Regole Aziendali}
	\label{sec:business_rules}
	
		\subsubsection{Regole di Vincolo}
		\subsubsection{Regole di Derivazione}
