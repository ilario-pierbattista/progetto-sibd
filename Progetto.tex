\documentclass[a4paper,twoside]{article}

%Packages
\usepackage{lmodern}
\usepackage{amsmath}	% advanced math symbols pkg
\usepackage{amsthm, amsthm}
\usepackage[utf8]{inputenc}	% utf8 encoding
\usepackage{graphicx} % pictures support
\usepackage{marginnote} % margin notes support
\usepackage{longtable} % table on multiple pages support
\usepackage[italian]{babel} % language support
\usepackage{enumitem} % custom enumeration support
\usepackage{rotating} % Rotating pictures support
\usepackage{float} % Support for floated images

\theoremstyle{definition}
\newtheorem{example}{Esempio}[section]


\makeindex

%Document begin
\begin{document}

\title{Progettazione di una Base di Dati per un'Offinica Meccanica}
\author{Gruppo 809\\
	Ilario Pierbattista\\
	Alessandro Staffolani\\
	Luka Petrovic}
\date{\today{}}
\maketitle
\newpage

\tableofcontents

% Introduzione
\section{Introduzione}

	Consci dell'attuale situazione economica del nostro Paese e dell'importanza del ruolo che le piccole e medie imprese rivestono nell'economia locale\footnote{Province di Fermo, Macerata ed Ascoli Piceno}, siamo oltremodo convinti che, per combattere questo periodo di profonda crisi, l'innovazione tecnologica dei processi produttivi e dei sistemi informativi delle aziende sia un ingrediente fondamentale. Troppo spesso, nelle piccole realtà imprenditoriali, questo aspetto viene ignorato, comportando in molti casi enormi sprechi in termini di risorse temporali che potrebbero essere evitati.
	
	Abbiamo scelto di svillupare una base di dati per un'officina meccanica. È solamente una delle tante realtà imprenditoriali che il nostro territorio ospita, l'abbiamo scelta per la disponibità di contatti diretti con un professionista del settore.
	
	Lo scopo di questo elaborato è quello di tenere traccia delle fasi di sviluppo di questo progetto e quello di fornire una documentazione adeguata sulla base di dati.
\newpage

% Sezione sull'analisi dei requisiti
%!TEX root = Progetto.tex
\section{Analisi dei Requisiti}
\label{sec:req_analysis}

	Nessun elemento del team conosceva direttamente la realtà imprenditoriale di un'officina meccanica, ma abbiamo dei contatti con un professionista al quale abbiamo chiesto informazioni. 
	
	Per capire quali sono i requisiti della base di dati, abbiamo raccolto informazioni attraverso il nostro contatto, quindi abbiamo proceduto a raffinare tali informazioni strutturandole in modo che risultino adeguate a procedere all'effettiva progettazione.
	
	\subsection{Raccolta delle Informazioni}
		
		La raccolta delle informazioni è stata effettuata attraverso un'intervista al nostro contatto e grazie ad alcuni documenti che egli stesso ci ha messo a disposizione. 
	
		\subsubsection{Intervista}
		Abbiamo intervistato il \emph{Sig. Adriano Staffonali}, titolare di un'officina meccanica nel comune di Treia (MC). L'intervista risale al 26 Ottobre 2014. Riportiamo, qui di seguito, i passaggi fondamentali.

		% Intervista		
		\begin{description}
			\item[A] 
				Di cosa si occupa la sua attività?
			\item[AS] 
				\emph{La mia attività è un'officina meccanica. Mi occupo di effettuare piccole e medie riparazioni di tipo meccanico ad autovetture e sono specializzato nella sostituzione, riparazione e manutenzione dei componenti elettronici. Inoltre la mia officina è autorizzata all'installazione di impianti a metano e GPL "}Landi Renzo\emph{", azienda leader nel settore al livello nazionale.}
			\item[A]
				Quante persone vi lavorano?
			\item[AS]
				\emph{Attualmente solo io, ma in passato ho avuto un paio di dipendenti.}
			\item[A]
				Come si articola una tipica giornata di lavoro?
			\item[AS]
				\emph{Solitamente ho sempre degli impianti da installare, che occupano la maggior parte della giornata. Ho un calendario dove segno tutte le scadenze a cui devo tener fede. Quando arriva un cliente, che abbia bisogno di una riparazione all'auto o dell'installazione di un impianto, devo fornirgli un preventivo. Se accetta, controllo quali pezzi devo acquistare, rintraccio i fornitori e li ordino.}
			\item[A]
				Che tipo di clienti sono i suoi? Privati? Aziende? Come tiene traccia dei loro dati?
			\item[AS]
				\emph{Per lo più i miei clienti sono privati, ma mi capita di lavorare con aziende e - occasionalmente - anche con enti pubblici. Tengo traccia solamente dei clienti quando effettuano nuovi impianti, in quanto la} Landi Renzo \emph{richiede per ogni nuovo cliente una scheda d'installazione da compilare on-line contenente dati anagrafici, recapiti e dati  dell'autovettura.}
			\item[A]
				Ammesso di avere individuato il guasto e di aver ben presente quali sono i pezzi da sostituire, solitamente, quanto sono precisi i preventivi per una riparazione? E quelli per l'installazione di un impianto?
 			\item[AS]
 				\emph{Per quanto riguarda le riparazioni, non si può dare sempre un preventivo preciso. Bisogna tener conto di alcuni aspetti: l'uso di pezzi di ricambio originali o meno e le ore di lavoro necessarie per effettuare la riparazione (di cui è sempre difficile effettuare previsioni precise). Per quanto riguarda l'installazione di impianti, invece, l'azienda che li produce e me li fornisce, predispone un listino prezzi completo che mi permette di effettuare preventivi in modo veloce e accurato.}
 			\item[A]
 				Non tiene uno storico delle riparazioni effettuate al fine di riutilizzare i dati per trovare soluzioni più velocemente in futuro?
 			\item[AS]
 				\emph{Uno storico no. Ho alcuni schemi tecnici che mi aiutano a risolvere il problema più velocemente. Però uno storico sarebbe utile.}
 			\item[A]
	 			Cosa appunta in questi schemi?
	 		\item[AS]
		 		\emph{Una breve descrizione del malfunzionamento riscontrato, la causa principale del malfunzionamento, una lista con i pezzi che comunemente bisogna sostituire per eliminare il malfunzionamento e qualche appunto sul procedimento da seguire.}
		 	\item[A]
			 	Come identifica i componenti di ricambio necessari?
			\item[AS]
				\emph{Dipende dal componente. Alcuni, come le bombole per il metano, non vengono scelti in base al modello dell'auto, ma in base alle dimensioni e alla loro capacità. Altri invece dipendono dal modello dell'automobile, che siano originali o compatibili. Altre volte ancora il modello dell'automobile non è sufficiente, visto tra esemplari dello stesso modello alcuni pezzi possono cambiare. In quel caso faccio riferimento al sito del produttore dell'auto, facendo una ricerca in base al numero del telaio.}
 			\item[A]
 				Per quanto riguarda i pagamenti da parte dei clienti, come si è organizzato? Inoltre, permette pagamenti dilazionati o rateizzati da parte dei clienti, che essi siano privati od aziende?
 			\item[AS]
 				\emph{Al momento utilizzo un archivio cartaceo per quanto riguarda fatture e ricevute. Pagamenti dilazionati? Raramente. Solitamente i miei clienti mi lasciano un acconto iniziale, quando la cifra del preventivo è considerevole, alla fine del lavoro pagano il resto. Ad alcune aziende, con le quali intrattengo rapporti frequentemente, permetto di effettuare pagamenti dilazionati. Quando si tratta invece di enti pubblici (ho avuto in passato rapporti commerciali con il comune di Treia) il pagamento dilazionato è l'unica soluzione.}
 			\item[A]
 				E per quanto riguarda i suoi fornitori? Le permettono pagamenti dilazionati?
 			\item[AS]
 				\emph{A dire il vero, raramente. Essendo la mia una piccola azienda, solo alcuni fornitori con cui ho instaurato un rapporto di fiducia nel tempo, mi permettono pagamenti dilazionati.}
 			\item[A]
 				Quindi lei si occupa da solo anche di tutta la contabilità, giusto?
 			\item[AS]
 				\emph{Non del tutto. Ho un commercialista. Lui si occupa di stilare il Bilancio e lo Stato Patrimoniale.}
 			\item[A]
 				Lei è solito tenere in magazzino pezzi per alcune riparazioni frequenti?
 			\item[AS]
 				\emph{Sì, cerco di avere sempre disponibili i pezzi fondamentali.}
 			\item[A]
 				Riesce a gestire adeguatamente il magazzino? Le è mai capitato di avere avuto dei prodotti che, soggetti magari all'usura del tempo, si siano rovinati?
 			\item[AS]
 				\emph{Ci sono alcuni prodotti che sono più soggetti di altri all'usura del tempo, altri invece che diventano obsoleti. Faccio un inventario completo delle rimanenze in magazzino una volta all'anno ed è un'attività che porta via molto tempo. Inoltre, quando utilizzo un pezzo per una riparazione, non vado ad aggiornare l'inventario, quindi non riesco a sapere ogni volta con precisione lo stato del magazzino.}
 			\item[A]
 				Per quanto riguarda i dati dei fornitori come ne tiene traccia? È sempre in grado di ritrovarli facilmente e immediatamente?
 			\item[AS]
 				\emph{Sinceramente no, non di tutti i dati. Se ne occupa il mio commercialista. Io ho solamente una rubrica cartacea con i numeri di telefono. Infatti, quando ho bisogno di dati che non siano i semplici numeri telefonici, devo contattare lui. Alcuni fornitori mi inviano le loro fatture via e-mail e queste contengono i dati dell'azienda di riferimento, ma anche in questo caso non è sempre agevole ritrovarli quando servono.}
 			\item[A]
 				Lei lavora da solo, ma ha detto di aver avuto un dipendente. Che tipo di contratto aveva? Si occupava personalmente delle buste paga?
 			\item[AS]
 				\emph{Giornalmente segnavo le sue ore di lavoro, quindi gli versavo l'importo a fine mese. Riuscivo ad occuparmene tranquillamente, era un solo dipendente d'altronde, ma se in futuro avessi bisogno di assumere più di una persona, dovrei adottare un altro metodo.}
		\end{description}
				
		\subsubsection{Documenti raccolti}
			Aggiungere le immagini dei documenti raccolti
			
		\subsubsection{Analisi dei processi interni}
			
			Abbiamo realizzato uno schema informale (figura \ref{fig:internal_processes}) che descrive il flusso dei dati all'effettuarsi delle procedure tipiche dell'attività.
			
			
			\begin{sidewaysfigure}
				\centering
				\includegraphics[width=22cm]{images/internal_processes.png}
				\caption{Analisi dei Processi Interni}
				\label{fig:internal_processes}
			\end{sidewaysfigure}
		
	\subsection{Requisiti Espressi nel Linguaggio Naturale}
	
		A partire dall’analisi dell’intervista e dall’analisi dei documenti in nostro possesso, abbiamo elaborato quelli che sono, a nostro avviso, i requisiti della base di dati che andremo a sviluppare.
		
		Il nostro obbiettivo è quello di sviluppare una base di dati per la gestione di un’officina meccanica di piccole medie dimensioni specializzata nell’installazione di impianti a metano e a GPL, ma che effettua anche riparazioni di natura meccanica ed elettronica alle autovetture.
		
		Bisognerà gestire i dati riguardanti i clienti e le loro autovetture, quelli riguardanti i fornitori e dei dipendenti. Bisognerà tenere traccia dei componenti presenti in magazzino, degli ordini effettuati e delle forniture ricevute.
		Si vuole tenere traccia dei dati riguardanti i preventivi emessi dall'attività e affiancandoli ai dati riguardanti le prestazioni effettuate a capo di tali preventivi, fornendo così uno storico consultabile delle attività effettuate nel tempo dall'azienda. Con il passare del tempo, tale storico diventerà una valida risorsa da cui attingere per agevolare il processo di formulazione dei preventivi, nonchè per rendere questi ultimi più precisi.
		Si vogliono conoscere i componenti più utilizzati nelle riparazioni e nelle installazioni, al fine di stabilire dei quantitativi minimi per ciascuno di essi da avere sempre a disposizione nel magazzino. Inoltre, si vuole fare in modo di evitare gli sprechi dovuti a componenti che diventano obsoleti o che si rovinano a causa dell'usura.
		Si vuole anche tenere traccia delle transazioni monetarie entranti (pagamenti dei clienti per le prestazioni ricevute) ed uscenti (versamenti ai fornitori ed ai dipendenti).
		
		Per quanto riguarda i clienti non dotati di partita iva, si vogliono conoscere il codice fiscale, il nome, il cognome, l’indirizzo di residenza, i vari recapiti. Per i clienti forniti di partita iva, si vuole tener traccia, appunto, della partita iva, della ragione sociale e dell’indirizzo della sede legale. 
		Nel caso in cui un cliente non dotato di partita iva richieda l’installazione di un nuovo impianto, sarà necessario conoscere anche il codice identificativo del documento di identità per le comunicazioni con la Motorizzazione Civile.
		
		Per quanto riguarda le autovetture sarà necessario conoscere la targa, la marca, il nome del modello, il numero del telaio. Se per un’autovettura viene richiesta l’installazione di un impianto, sarà necessario conoscere anche la cilindrata, l’anno di immatricolazione e la data dell’ultima revisione. Ad installazione eseguita sarà necessario aggiungere anche la data in cui l'impianto viene collaudato.
		
		A proposito dei fornitori, sarà necessario conoscerne la partita via, la ragione sociale, i vari recapiti, i tempi medi di consegna, la modalità di pagamento preferita (bonifico bancario o assegno) ed - eventualmente - il codice IBAN.
		
		Dei dipendenti si vuole tener traccia di codice fiscale, nome, cognome, luogo di nascita, data di nascita, indirizzo di residenza, retribuzione oraria (ove il dipendente venga pagato in base alle ore di lavoro effettuate), stipendio mensile (ove invece il dipendente abbia un contratto che prevede una retribuzione mensile costante), modalità di riscossione ed - eventualmente - il codice IBAN. Si vogliono anche conoscere le presenze che i dipendenti effettuano, tenendo conto dell’ora di inizio del turno, l’ora di fine e la data di rifermento. 
		L'inserimento degli orari di inizio e di fine del turno può essere effettuata manualmente dal titolare alla fine della giornata oppure tramite l'installazione di un dispositivo di lettura di badge magnetici.
		
		Riguardo ai componenti si vogliono conoscere il nome, il tempo di validità dal momento dell'acquisto (dopo il quale il componente risulta rovinato dall'usura o diviene obsoleto), il prezzo di vendita e la quantità minima che deve essere sempre presente nel magazzino.
		
		L'acquisto dei componenti viene formalizzato attraverso un ordine. Un ordine sarà composto da una o più forniture, dalla partita iva del fornitore presso cui si fa l'ordine, dalla data in cui l'ordine viene effettuato e dalla data in cui è stata consegnata la merce.
		
		Per fornitura si intende un insieme omogeneo di componenti acquistati nello stesso ordine. Ognuna di esse sarà caratterizzata dal componente, dalla quantità acquistata e dal prezzo unitario di acquisto.
		
		Per quanto riguarda la gestione del magazzino, al fine di evitare che alcuni componenti diventino obsoleti o si rovinino con il tempo, bisogna fare in modo che vengano utilizzati prima quelli la cui \emph{data di scadenza} è più prossima di altri. Per fare ciò, conoscendo il periodo di validità del componente, sarà necessario avere anche la data d'acquisto.
		Possiamo considerare il magazzino come un elenco di \emph{forniture attive}, ovvero una fornitura i cui componenti non siano già stati tutti utilizzati. Affiancando alla fornitura di riferimento, la quantità rimanente dei componenti di quella fornitura, avremo tutti i dati necessari.
		
		Riguardo i preventivi sarà necessario conoscere innanzi tutto la categoria dell'intervento richiesto (riparazione, installazione di un impianto a metano, installazione di un impianto a gpl, collaudo, revisione). Serviranno inoltre la data di emissione del preventivo, la data in cui dovrebbe cominciare il lavoro, i componenti che si prevede saranno utilizzati per compiere il lavoro, la stima dei costi della manodopera, la stima dei costi di eventuali servizi aggiuntivi e l'ammontare di un eventuale acconto versato dal cliente.
		Nel caso in cui il cliente necessiti di una riparazione sarà utile aggiungere una brevissima descrizione dei sintomi.
		Se il cliente necessita dell’installazione di un nuovo impianto, sarà necessario tenere conto anche della tipologia del sistema di alimentazione (iniezione o aspirazione). Inoltre, per avere la completa compatibilità con il modello per i preventivi imposti dall'azienda \emph{Landi Renzo}, di ogni componente necessario per effettuare l'installazione di un impianto, sarà necessario conoscere la relativa ubicazione nell'autovettura, facendo distinzione tra i componenti necessari per il vano motore e quelli per il vano bagagliaio.
		
		Riguardo le prestazioni, eseguite a fronte di un preventivo, si vuole conoscere il preventivo di riferimento, i tempi effettivi di esecuzione, la data in cui è stato finito il lavoro, i componenti che sono stati effettivamente utilizzati, costo manodopera, il costo di eventuali servizi aggiuntivi, i lavoratori che hanno le hanno eseguite.
		Nel caso di riparazioni è utile aggiungere anche una brevissima descrizione del danno riscontrato ed una descrizione più approfondita sul procedimento utilizzato per effettuare la riparazione.
		
		Ad ogni prestazione fa capo una fattura. I dati delle fatture di cui è importante tener traccia sono il numero progressivo di fattura (da azzerare all'inizio di ogni anno), la data di emissione, l'ammontare dell'imponibile, l'ammontare delle imposte, ammontare di un eventuale sconto, ammontare di eventuali incentivi il sistema di pagamento (rimessa diretta o rimessa differita), il tipo di pagamento (assegno, bonifico o contanti), lo stato del pagamento. Nel caso di pagamenti con rimessa differita sarà necessario conoscere anche la data di scadenza.
		
		Si vogliono conoscere anche i dati relativi alle transazioni monetarie, entranti o uscenti che siano. Nel dettaglio, si vuole tener traccia della quota delle singole transazioni e della data di emissione.
		
	\subsection{Glossario dei Termini}
	
		Al fine di evitare la presenza di ambiguità, abbiamo stilato un glossario dei termini più importanti a cui faremo riferimento di qui in avanti.
						
		\begin{longtable}{| p{2.5cm} | p{4.5cm} | p{2cm} | p{2.5cm} |}
				
			\hline
			\textbf{Termine} & 
			\textbf{Descrizione} & 
			\textbf{Sinonimi} & 
			\textbf{Collegamenti} \\
			
			\endfirsthead
				
			\hline
			\textbf{Termine} & 
			\textbf{Descrizione} & 
			\textbf{Sinonimi} & 
			\textbf{Collegamenti} \\
			
			\endhead
			
			\hline
			Cliente & 
			Persona fisica o giuridica che abbia avuto rapporti con l'azienda.
			&&\\ \hline
			Autovettura &
			Automobile di un cliente che debba subire o abbia già subito un intervento da parte dei lavoratori dell’azienda. &
			Automobile, Veicolo &
			Cliente, Preventivo
			\\ \hline
			Preventivo &
			Stima dei costi, dei tempi e dei componenti necessari relativi all’esecuzione di un intervento su un’autovettura. & &
			Autovettura, Componente, Prestazione
			\\ \hline
			Preventivo di Riparazione &
			Preventivo relativo alla riparazione di un guasto di un’autovettura & &
			Preventivo 
			\\ \hline
			Preventivo d'Installazione & 
			Preventivo relativo all’installazione di un nuovo impianto in un’autovettura. & &
			Preventivo
			\\ \hline
			Componente &
			Qualsiasi oggetto fisico necessario alla corretta esecuzione di una riparazione o di una installazione di un impianto su di un’autovettura &
			Prodotto &
			Fornitore, Prestazione, Preventivo
			\\ \hline
			Fornitore & 
			Azienda che abbia fornito all’officina qualsiasi tipo di componente necessario. & &
			Componente, Fornitura
			\\ \hline
			Ordine &
			Insieme di componenti acquistati presso un fornitore. I componenti omogenei sono organizzati in forniture. &
			&
			Componente, Fornitura
			\\ \hline
			Fornitura & 
			Insieme omogeneo di componenti acquistati presso un fornitore nel medesimo ordine. &
			&
			Componente, Fornitore
			\\ \hline
			Magazzino &
			Insieme totale dei componenti depositati fisicamente in una apposita area nei locali utilizzati dall'attività in attesa di essere utilizzati. &
			Deposito &
			Componente, Fornitura Attiva
			\\ \hline
			Fornitura Attiva &
			Si fa riferimento a quelle forniture i cui componenti, totalmente o in parte, sono ancora presenti in magazzino in attesa di essere utilizzati. &
			&
			Componente, Fornitura, Magazzino
			\\ \hline
			Data di scadenza &
			Riferito ad un componente è la data, calcolata a partire da quella d'acquisto, oltre il quale il componente diventa inutilizzabile per obsolescenza o per usura. &
			&
			Componente
			\\ \hline
			Dipendente & 
			Persona fisica che abbia lavorato per l’officina &
			Lavoratore, Operatore &
			\\ \hline
			Transazione &
			Flusso di denaro uscente o entrante nella cassa dell’attività. &
			Flusso di Cassa &
			\\ \hline
			Prestazione &
			Attività eseguita dai lavoratori dell’officina su di un’autovettura. & &
			Preventivo
			\\ \hline
			Sintomo &
			Malfunzionamento direttamente verificabile di un autoveicolo, individuabile senza il bisogno di conoscerne le cause. &
			& \\ \hline
			Pagamento &
			Transazione di denaro entrante a seguito di una prestazione fornita ad un cliente. &&
			Prestazione, Transazione
			\\ \hline
			Versamento &
			Transazione di denaro uscente a seguito di una fornitura ricevuta o di uno stipendio versato ad un dipendente. & 
			Spesa & 
			Transazione, Fornitore, Dipendente 
			\\ \hline
			Collaudo &
			Attività relativa alla verifica specifica del corretto funzionamento del serbatoio installato con il nuovo impianto (che sia a metano o gpl). &&
			Autovettura
			\\ \hline
			Revisione & 
			Attività di verifica del corretto funzionamento dei tutte le parti dell’autovettura. &&
			Autovettura
			\\ \hline
			Retribuzione Oraria & 
			Ammontare della retribuzione di un dipendente per ogni ora di lavoro. && 
			Dipendente
			\\ \hline
			Modalità di Riscossione &
			Modalità, indicata dal dipendente, con cui quest’ultimo riceve lo stipendio. &&
			Dipendente 
			\\ \hline
			Impianto & 
			Infrastruttura di alimentazione di un’autovettura.
			&&\\ \hline
			Imponibile &
			Somma di denaro su cui vanno calcolate le imposte previste per legge. &&
			Pagamento 
			\\ \hline
			Rimessa diretta &
			Si intende il sistema di pagamento nel quale la prestazione viene pagata immediatamente dopo la consegna della fattura &
			& \\ \hline
			Rimessa differita &
			Si intende il sistema di pagamento nel quale la prestazione viene pagata dal cliente entro 30, 60 o 90 giorni dalla consegna della fattura. &
			& \\ \hline
			Tipo di pagamento &
			Modalità di trasferimento di denaro. &
			& \\ \hline
				
		\end{longtable}
		
	\subsection{Eliminazione delle Ambiguità Presenti}
		
		Potrebbe risultare ambiguo l'utilizzo che viene fatto del termine \emph{componente}. Ci si riferisce, con componente, ad un oggetto fisico necessario all'esecuzione di una prestazione, ma viene usato anche per identificare la classe stessa dell'oggetto, piuttosto che l'oggetto singolo.
		
		Si propone la seguente precisazione:
		\begin{description}
			\item[Componente]
				Classe di oggetti reali necessari ad effettuare una prestazione.
			\item[Articolo]
				Oggetto fisico necessario ad effettuare una prestazione.
		\end{description}
		
		\begin{example}
			Nel magazzino ci sono 10 bombole da 50 litri. Si dirà che nel magazzino sono presenti 10 articoli del componente "bombola da 50 litri".
		\end{example}
		
		Apporteremo, ove necessario, le correzioni nella sezione successiva.
		
	\subsection{Strutturazione dei Requisiti}
	
		\subsubsection{Frasi di Carattere Generale}
					
			Bisognerà gestire i dati riguardanti i clienti e le loro autovetture, quelli riguardanti i fornitori e dei dipendenti. Bisognerà tenere traccia dei componenti presenti in magazzino, degli ordini effettuati e delle forniture ricevute.
			Si vuole tenere traccia dei dati riguardanti i preventivi emessi dall'attività e affiancandoli ai dati riguardanti le prestazioni effettuate a capo di tali preventivi, fornendo così uno storico consultabile delle attività effettuate nel tempo dall'azienda. Con il passare del tempo, tale storico diventerà una valida risorsa da cui attingere per agevolare il processo di formulazione dei preventivi, nonchè per rendere questi ultimi più precisi.
			Si vogliono conoscere i componenti più utilizzati nelle riparazioni e nelle installazioni, al fine di stabilire dei quantitativi minimi per ciascuno di essi da avere sempre a disposizione nel magazzino. Inoltre, si vuole fare in modo di evitare gli sprechi dovuti a componenti che diventano obsoleti o che si rovinano a causa dell'usura.
			Si vuole anche tenere traccia delle transazioni monetarie entranti (pagamenti dei clienti per le prestazioni ricevute) ed uscenti (versamenti ai fornitori ed ai dipendenti).
			
			L'acquisto dei componenti viene formalizzato attraverso un ordine.
			
		\subsubsection{Frasi relative ai Clienti}
		
			Per quanto riguarda i clienti non dotati di partita iva, si vogliono conoscere il codice fiscale, il nome, il cognome, l’indirizzo di residenza, i vari recapiti. Per i clienti forniti di partita iva, si vuole tener traccia, appunto, della partita iva, della ragione sociale e dell’indirizzo della sede legale. 
			Nel caso in cui un cliente non dotato di partita iva richieda l’installazione di un nuovo impianto, sarà necessario conoscere anche il codice identificativo del documento di identità per le comunicazioni con la Motorizzazione Civile.
		
		\subsubsection{Frasi relative alle Autovetture}
			
			Per quanto riguarda le autovetture sarà necessario conoscere la targa, la marca, il nome del modello, il numero del telaio. Se per un’autovettura viene richiesta l’installazione di un impianto, sarà necessario conoscere anche la cilindrata, l’anno di immatricolazione e la data dell’ultima revisione. Ad installazione eseguita sarà necessario aggiungere anche la data in cui l'impianto viene collaudato.
		
		\subsubsection{Frasi relative ai Fornitori}
			
			A proposito dei fornitori, sarà necessario conoscerne la partita via, la ragione sociale, i vari recapiti, i tempi medi di consegna, la modalità di pagamento preferita (bonifico bancario o assegno) ed - eventualmente - il codice IBAN.
		
		\subsubsection{Frasi relative ai Dipendenti} 
			
			Dei dipendenti si vuole tener traccia di codice fiscale, nome, cognome, luogo di nascita, data di nascita, indirizzo di residenza, retribuzione oraria (ove il dipendente venga pagato in base alle ore di lavoro effettuate), stipendio mensile (ove invece il dipendente abbia un contratto che prevede una retribuzione mensile costante), modalità di riscossione ed - eventualmente - il codice IBAN. Si vogliono anche conoscere le presenze che i dipendenti effettuano, tenendo conto dell’ora di inizio del turno, l’ora di fine e la data di rifermento. 
			L'inserimento degli orari di inizio e di fine del turno può essere effettuata manualmente dal titolare alla fine della giornata oppure tramite l'installazione di un dispositivo di lettura di badge magnetici.
			
		\subsubsection{Frasi relative ai Componenti}
			
			Riguardo ai componenti si vogliono conoscere il nome, il tempo di validità dal momento dell'acquisto (dopo il quale il componente risulta rovinato dall'usura o diviene obsoleto), il prezzo di vendita e la quantità minima che deve essere sempre presente nel magazzino.
			
		\subsubsection{Frasi relative agli Ordini}
		
			L'acquisto dei componenti viene formalizzato attraverso un ordine. Un ordine sarà composto da una o più forniture, dalla partita iva del fornitore presso cui si fa l'ordine, dalla data in cui l'ordine viene effettuato e dalla data in cui è stata consegnata la merce.
		
		\subsubsection{Frasi relative alle Forniture}
		
			Per fornitura si intende un insieme di articoli dello stesso componente acquistati nello stesso ordine. Ognuna di esse sarà caratterizzata dal componente, dalla quantità acquistata e dal prezzo unitario di acquisto.
			
		\subsubsection{Frasi relative al Magazzino}
			
			Per quanto riguarda la gestione del magazzino, al fine di evitare che alcuni articoli diventino obsoleti o si rovinino con il tempo, bisogna fare in modo che vengano utilizzati prima quelli la cui \emph{data di scadenza} è più prossima di altri. Per fare ciò, conoscendo il periodo di validità del componente, sarà necessario avere anche la data d'acquisto dell'articolo.
			Possiamo considerare il magazzino come un elenco di \emph{forniture attive}, ovvero una fornitura i cui articoli non siano già stati tutti utilizzati. Affiancando alla fornitura di riferimento, la quantità rimanente degli articoli di quella fornitura, avremo tutti i dati necessari.
			
		\subsubsection{Frasi relative ai Preventivi}
		\label{sec:frasi_preventivi}
			
			Riguardo i preventivi sarà necessario conoscere innanzi tutto la categoria dell'intervento richiesto (riparazione, installazione di un impianto a metano, installazione di un impianto a gpl, collaudo, revisione). Serviranno inoltre la data di emissione del preventivo, la data in cui dovrebbe cominciare il lavoro, i componenti che si prevede saranno utilizzati per compiere il lavoro, la stima dei costi della manodopera, la stima dei costi di eventuali servizi aggiuntivi e l'ammontare di un eventuale acconto versato dal cliente.
			Nel caso in cui il cliente necessiti di una riparazione sarà utile aggiungere una brevissima descrizione dei sintomi.
			Se il cliente necessita dell’installazione di un nuovo impianto, sarà necessario tenere conto anche della tipologia del sistema di alimentazione (iniezione o aspirazione). Inoltre, per avere la completa compatibilità con il modello per i preventivi imposti dall'azienda \emph{Landi Renzo}, di ogni componente necessario per effettuare l'installazione di un impianto, sarà necessario conoscere la relativa ubicazione nell'autovettura, facendo distinzione tra i componenti necessari per il vano motore e quelli per il vano bagagliaio.
			
		\subsubsection{Frasi relative alle Prestazioni}
		
			Riguardo le prestazioni, eseguite a fronte di un preventivo, si vuole conoscere il preventivo di riferimento, i tempi effettivi di esecuzione, la data in cui è stato finito il lavoro, i componenti che sono stati effettivamente utilizzati, costo manodopera, il costo di eventuali servizi aggiuntivi, i lavoratori che hanno le hanno eseguite.
			Nel caso di riparazioni è utile aggiungere anche una brevissima descrizione del danno riscontrato ed una descrizione più approfondita sul procedimento utilizzato per effettuare la riparazione.
			
		\subsubsection{Frasi relative alle Fatture}
		
			Ad ogni prestazione fa capo una fattura. I dati delle fatture di cui è importante tener traccia sono il numero progressivo di fattura (da azzerare all'inizio di ogni anno), la data di emissione, l'ammontare dell'imponibile, l'ammontare delle imposte, ammontare di un eventuale sconto, ammontare di eventuali incentivi il sistema di pagamento (rimessa diretta o rimessa differita), il tipo di pagamento (assegno, bonifico o contanti), lo stato del pagamento. Nel caso di pagamenti con rimessa differita sarà necessario conoscere anche la data di scadenza.
		
		\subsubsection{Frasi relative alle Transazioni}
		
			Si vogliono conoscere anche i dati relativi alle transazioni monetarie, entranti o uscenti che siano. Nel dettaglio, si vuole tener traccia della quota delle singole transazioni e della data di emissione.
			
	\subsection{Specifica delle Operazioni}
		
		Abbiamo individuato le operazioni che, con lo sviluppo di tale base di dati, si intendono svolgere sulla stessa. Abbiamo aggiunto ad ogni operazione una stima della frequenza con la quale l'operazione stessa viene effettuata.
		
		Non sono presenti operazioni di cancellazione. Quest'ultime infatti sono previste solo nel caso in cui vengano commessi degli errori in fase di inserimento.
	
		% Impostare un separatore custom
		\setenumerate[1]{label=OP\arabic*}
		\begin{enumerate}
			
			% Inserimenti
			\item \label{op:new_cliente} Inserimento di un nuovo cliente (3 volte a settimana)
			\item \label{op:new_privato} Inserimento di un nuovo cliente non dotato di partita iva (2 volte a settimana)
			\item \label{op:new_azienda} Inserimento di un nuovo cliente dotato di partita iva (1 volta a settimana)
			\item \label{op:new_auto} Inserimento di una nuova autovettura (3 volte a settimana)
			\item \label{op:new_fornitore} Inserimento di un nuovo fornitore (4 volte all’anno)
			\item \label{op:new_componente} Inserimento di un nuovo componente (2 volte al mese)
			\item \label{op:new_ordine} Inserimento di un nuovo ordine (2 volte a settimana)
			\item \label{op:new_fornitura} Inserimento di una nuova fornitura (10 volte a settimana)
			\item \label{op:new_preventivo} Inserimento di un nuovo preventivo (10 volte a settimana)
			\item \label{op:new_prestazione} Inserimento di una nuova prestazione (10 volte a settimana)
			\item \label{op:new_fattura} Inserimento di una nuova fattura (10 volte a settimana)
			\item \label{op:new_operatore} Inserimento di un nuovo dipendente (1 volta all’anno)
			\item \label{op:new_recapito}Inserimento di un nuovo recapito (2 volta per ogni nuovo cliente, 3 volte per ogni dipendente, in media 4 volte per ogni nuovo fornitore)
			\item \label{op:new_presenza}Inserimento di una nuova presenza (1 volta al giorno per ogni dipendente)
			\item \label{op:new_transazione}Inserimento di una nuova transazione (1 volta ogni 2 preventivi, 1 volta al mese per ogni dipendente, 1 volta per ogni fornitura, 1 volta per ogni prestazione)
			
			% Assegnazioni
			\item \label{op:ass_componente_preventivo}Assegnazione di un componente ad un preventivo (6 volte per ogni preventivo)
			\item \label{op:ass_componente_prestazione}Assegnazione di un componente ad una prestazione (6 volte per ogni prestazione)
			\item \label{op:ass_fornitura_ordine}Assegnazione di una fornitura ad un ordine (5 volte per ogni ordine)
			\item \label{op:ass_fornitura_magazzino} Assegnazione di una fornitura al magazzino a seguito della consegna di un ordine (5 volte per ogni ordine)
			
			% Aggiornamento
			\item \label{op:edit_cliente}Modifica dei dati di un cliente (30 volte l’anno)
			\item \label{op:edit_fornitore}Modifica dei dati di un fornitore (2 volte l’anno)
			\item \label{op:edit_operatore}Modifica dei dati di un dipendente (1 volta l’anno)
			\item \label{op:edit_componente}Modifica del prezzo di vendita di un componente (5 volte al mese)
			\item \label{op:reg_ordine} Registrazione dell'arrivo di un ordine (2 volte a settimana)
			
			% Consultazione
			\item \label{op:show_collaudo}Consultazione della data dell’ultimo collaudo per un’autovettura (1 volte a settimana)
			\item \label{op:show_revisione}Consultazione della data dell’ultima revisione per un’autovettura (1 volte a settimana)
			\item \label{op:show_fattura}Consultazione dei dati per stilare una fattura o una ricevuta fiscale (10 volte a settimana)
			\item \label{op:show_transazioni}Consultazione delle transazioni avvenute in un certo periodo (1 volta a settimana)
			\item \label{op:show_riparazioni}Consultazione dello storico delle riparazioni (2 volte al giorno)
			\item \label{op:show_preventivi}Consultazione dello storico dei preventivi (2 volte al giorno)
			\item \label{op:check_componente}Consultazione della disponibilità di un componente (4 volte al giorno)
			\item \label{op:check_turni}Consultazione delle presenze di un dipendente in un arco temporale (1 volta al mese)
			\item \label{op:list_componenti}Consultazione della lista dei componenti presenti (1 volta a settimana)
			\item \label{op:stats_componenti}Consultazione della lista dei componenti più usati (2 volte al mese)
			\item \label{op:tobuy_componenti}Consultazione della lista dei componenti che si dovrebbero acquistare nuovamente (1 volta a settimana)
			\item \label{op:show_recapiti_cliente}Consultazione della lista dei recapiti per un cliente (10 volte a settimana)
			\item \label{op:show_recapiti_fornitore}Consultazione della lista dei recapiti per un fornitore (2 volte a settimana)
			\item \label{op:show_recapiti_operatore}Consultazione della lista dei recapiti per un dipendente (1 volta a settimana)
			\item \label{op:list_fatture_pending}Consultazione della lista delle fatture che devono essere ancora pagate (1 volta al giorno)
			\item \label{op:list_ordini_pending}Consultazione della lista degli ordini che devono ancora arrivare (1 volta a settimana)
			\item \label{op:todo_list}Consultazione della lista dei lavori da eseguire (2 volta al giorno)
			\item \label{op:calc_stipendio_operatore}Calcolo dello stipendio per un dipendente (1 volta al mese)
			\item \label{op:stats_prevetivi_prestazioni}Consultazione delle statistiche riguardanti lo scostamento tra i costi preventivati e i costi effettivi delle prestazioni, con la possibilità di scomporre le voci dei costi tra costi dei componenti, costi della manodopera e costi dei servizi aggiuntivi (1 volta a settimana).
			\item \label{op:stats_costi}Consultazione delle statistiche riguardanti la variazione dei costi di un componente (1 volta a settimana)
			
		\end{enumerate}

\newpage

% Sezione sulla progettazione concettuale
%!TEX root = Progetto.tex
\section{Progettazione Concettuale}
	
	\subsection{Strategia di Progetto}
		
		La strategia di progettazione che ci sembra più adatta per svolgere il nostro lavoro è la strategia \emph{Inside-Out}. Partiremo dai concetti più rilevanti, emersi in fase di analisi dei requisiti, e da questi 
						
		{\color{red}{Questa sezione va ampliata e corretta}}
		
	\subsection{Individuazione dello Scheletro dello Schema ER}
		
		Dalle specifiche che abbiamo formulato risulta che uno dei punti fondamentali da affrontare è quello della memorizzazione dei preventivi emessi dall'attività.
		Ad ogni \emph{Prestazione} effettuata, corrisponde un \emph{Preventivo} precedentemente emesso.
		
		\begin{figure}[H]
			\centering
			\includegraphics[width=9cm]{images/diagrams/preventivo_prestazione.png}
		\end{figure}
		
		Alla formulazione di ogni preventivo, si fa una stima dei componenti che si reputa saranno necessari per eseguire la prestazione preventivata. Non sempre tale previsione è completamente esatta, generalmente i componenti effettivamente utilizzati in una prestazione sono diversi da quelli previsti in un preventivo.
		L'associazione tra i \emph{Componenti} e il \emph{Preventivo} risulta immediata.
		
		\begin{figure}[H]
			\centering
			\includegraphics[width=9cm]{images/diagrams/preventivo_componente.png}
		\end{figure}
		
		Prima di esplicitare l'associazione tra le prestazioni eseguite ed i componenti effettivamente utilizzati, affrontiamo la questione degli ordini e del magazzino.
		
		L'acquisto di articoli presso un fornitore è formalizzato in un ordine, il quale, come da specifiche, è organizzato in più forniture, ovvero insiemi di articoli di uno stesso componente.
		Ad un \emph{Ordine} sono associate una o più \emph{Forniture} di articoli, ognuna delle quali fanno riferimento ad un \emph{Componente}.
		
		\begin{figure}[H]
			\centering
			\includegraphics[width=11.5cm]{images/diagrams/componente_ordine.png}
		\end{figure}
		
		Il magazzino, nella realtà, è composto dai vari articoli acquistati che sono in attesa di essere utilizzati. La classificazione degli articoli avviene, in primo luogo per componente, in secondo luogo per fornitura d'appartenenza. Non vi è così il bisogno di registrare ogni articolo individualmente, ma basterà riferirsi alle relative forniture.
		Il \emph{Magazzino} è una composizione di \emph{Forniture} i cui articoli sono depositati in attesa di essere utilizzati.
		
		\begin{figure}[H]
			\centering
			\includegraphics[width=9cm]{images/diagrams/magazzino_fornitura.png}
		\end{figure}
		
		Per identificare con precisione quali articoli sono stati utilizzati per l'esecuzione di una prestazione, sarà sufficiente riferirsi alla fornitura relativa agli stessi. Da questa si ottengono le informazioni sul componente (quindi il prezzo di vendita e il tempo di validità) e sulla data d'acquisto.
		Ad ogni utilizzo, si provvederà ad aggiornare le quantità rimanenti degli articoli delle forniture utilizzate.
		Per l'esecuzione di una \emph{Prestazione} si possono utilizzare gli articoli di più \emph{Forniture}.
		
		\begin{figure}[H]
			\centering
			\includegraphics[width=9cm]{images/diagrams/prestazione_fornitura.png}
		\end{figure}
		
		Passiamo alla questione degli operatori. Una prestazione viene eseguita da uno o più operatori. Di ogni operatore si vuole tener traccia dei turni di lavoro effettuati.
		Alla \emph{Prestazione}, saranno associati uno o più \emph{Operatori} ad ognuno dei quali sono associati i relativi \emph{Turni} di lavoro.
		
		\begin{figure}[H]
			\centering
			\includegraphics[width=9cm]{images/diagrams/operatore_turno_prestazione.png}
		\end{figure}
		
		Occupiamoci ora delle zone periferiche dello schema. Un preventivo viene effettuato quando un cliente richiede un intervento alla propria auto.
		Ad ogni \emph{Cliente} vengono associate una o più \emph{Autovetture}. Ogni \emph{Preventivo} si riferisce ad una specifica \emph{Autovettura}.
		
		\begin{figure}[H]
			\centering
			\includegraphics[width=9cm]{images/diagrams/cliente_autovettura_preventivo.png}
		\end{figure}
		
		Ogni \emph{Ordine} viene effettuato presso un \emph{Fornitore}.
		
		\begin{figure}[H]
			\centering
			\includegraphics[width=9cm]{images/diagrams/ordine_fornitore.png}
		\end{figure}
		
		Notiamo che \emph{Cliente} rappresenta sia \emph{Privati} che \emph{Aziende} (rispettivamente, clienti non dotati di partita iva e clienti dotati di partita iva).
		
		\begin{figure}[H]
			\centering
			\includegraphics[width=7.5cm]{images/diagrams/cliente.png}
		\end{figure}
		
		Inoltre \emph{Clienti}, \emph{Fornitori} ed \emph{Operatori} possono essere generalizzati dall'entità \emph{Persona}.
		
		\begin{figure}[H]
			\centering
			\includegraphics[width=10cm]{images/diagrams/persona.png}
		\end{figure}
		
		Ad ogni \emph{Persona} saranno associati uno o più \emph{Recapiti}.
		
		\begin{figure}[H]
			\centering
			\includegraphics[width=9cm]{images/diagrams/persona_recapito.png}
		\end{figure}
		
		Possiamo concludere lo sviluppo della struttura del diagramma ER affrontando la questione delle transazioni. Avviene una \emph{Transazione} ogni volta che viene versato un acconto per un \emph{Preventivo}, ogni volta che viene saldata la \emph{Fattura} di una \emph{Prestazione}, ogni volta che viene pagato un \emph{Ordine} ed ogni volta che viene pagato lo stipendio di un \emph{Operatore}.
		
		\begin{figure}[H]
			\centering
			\includegraphics[width=12cm]{images/diagrams/transazione.png}
		\end{figure}
			
		Il diagramma in figura \ref{fig:scheletro_er} rappresenta lo scheletro dello schema ER.
		
		\begin{sidewaysfigure}
			\centering
			\includegraphics[width=22cm]{images/diagrams/schema.png}
			\caption{Scheletro del diagramma ER}
			\label{fig:scheletro_er}
		\end{sidewaysfigure}
	
	\subsection{Sviluppo delle Componenti dello Schema}
	
		Ottenuto lo scheletro generale del diagramma ER procediamo ad esplicitare, delle entità principali, l'insieme degli attributi che ognuna di esse possiede.
		Una volta sviluppati gli attributi delle entità principali svilupperemo le relationship che le legano, raggruppandole tra loro sulla falsariga dei modelli elaborati allo step precedente.
		
		\begin{description}
			\item[NB]
				Ogni generalizzazione effettuata è da considerarsi totale.
		\end{description}
		
		\subsubsection{Persona}
		
			In figura \ref{fig:persona} troviamo lo sviluppo degli attributi dell'entità \emph{Persona} e delle relative entità che la estendono.
			
			L'identificativo dell'entità \emph{Persona} è costituito dall'attributo "Codice Fiscale o P.Iva", capace di identificare così sia privati che aziende. Ogni \emph{Persona} è caratterizzata anche dall'attributo composto "Indirizzo", sviluppabile in "Città", "Via", "Civico", "CAP", che identifica l'indirizzo di riferimento della persona stessa.
			
			Le entità figlie di \emph{Persona} sono \emph{Fornitore}, \emph{Operatore} e \emph{Cliente}. Quest'ultimo può essere ulteriormente scomposto in altre due entità figlie \emph{Privato} e \emph{Azienda}.
			
			\emph{Privato} possiede gli attributi "Nome", "Cognome" e "Numero Documento Identità", mentre per l'entità \emph{Azienda} si è reso necessario avere solamente l'attributo "Ragione Sociale".
			
			\emph{Fornitore} è dotato degli attributi "Ragione Sociale", "IBAN", "Tempi Consegna" (numero di giorni feriali necessari in media affinchè la merce ordinata al fornitore arrivi) e "Modalità Pagamento" (specifica la modalità di pagamento tra assegno e bonifico bancario).
			
			\emph{Operatore} ha gli attributi "Nome", "Cognome", "IBAN", "Stipendio" (ammontare dello stipendio mensile, se il lavoratore ha un contratto a retribuzione fissa), "Retribuzione oraria" (se il lavoratore ha un contratto che prevede uno stipendio calcolato in base alle ore di lavoro), "Modalità Riscossione" (specifica la modalità di riscossione dello stipendio tra assegno, bonifico o contanti\footnote{Applicabile solo nel caso in cui l'ammontare del pagamento non superi l'importo massimo a norma di legge. Attualmente il limite per i pagamenti in contanti ammonta a 1000.00\EUR.}) e "Dati Anagrafici" (attributo composto da "Data di Nascita", "Comune di Nascita", "Provinicia").
						
			\begin{figure}[H]
				\centering
				\includegraphics[width=12cm]{images/finitures/persona.png}
				\caption{Sviluppo di Persona}
				\label{fig:persona}
			\end{figure}
		
		\subsubsection{Autovettura}
			
			Continuiamo con gli attributi che caratterizzano l'entità \emph{Autovettura} (Diagramma in figura \ref{fig:autovettura})
			
			\begin{figure}[H]
				\centering
				\includegraphics[width=9cm]{images/finitures/autovettura.png}
				\caption{Sviluppo dell'entità Autovettura}
				\label{fig:autovettura}
			\end{figure}
			
			\emph{Autovettura} comprende gli attributi "Targa", "Marca", "Modello", "Telaio" (numero di serie del telaio che identifica univocamente un veicolo che viene inciso sul telaio del veicolo e viene indicato nel libretto di circolazione), "Ultima Revisione" e "Ultimo Collaudo" (rispettivamente le date in cui è stata effettuata la revisione dell'auto e il collaudo di un eventuale impianto di alimentazione differente da quello di fabbricazione), "Anno di Immatricolazione", "Cilindrata".
		
			Si è scelto l'attrubuto "Targa" come chiave primaria dell'entità piuttosto che l'attributo "Telaio" nonostante anche quest'ultimo identifichi univocamente l'autovettura. Riteniamo che sia più agevole identificare un'autovettura attraverso la targa poichè tale informazione è più facilmente reperibile rispetto al seriale del telaio.
			
		\subsubsection{Preventivo}
		
			In figura \ref{fig:preventivo}, il diagramma espone gli attributi dell'entità \emph{Preventivo}.
			
			\emph{Preventivo} è costituito dagli attributi "Codice" (identificativo numerico interno all'azienda del preventivo fornito), "Data Emissione", "Tempo Stimato" (ovvero la stima del numero di giorni necessari all'esecuzione del lavoro), "Data Inizio" (data in cui il lavoro è stato pianificato per essere eseguito), "Categoria" (riparazione, installazione di un impianto a metano, installazione di un impianto a gpl, collaudo o revisione), "Sistema Alimentazione" (attributo necessario per le installazioni di nuovi impianti, necessari a specificare il sistema di alimentazione tra sistema a iniezione e sistema ad aspirazione), "Sintomi" (ovvero una breve descrizione del malfunzionamento riscontrato, nel caso in cui si tratti di una riparazione), "Costo Servizi" (composizione della stima dei costi dei servizi aggiuntivi e della manodopera).
			
			\begin{figure}[H]
				\centering
				\includegraphics[width=12cm]{images/finitures/preventivo.png}
				\caption{Sviluppo dell'entità Preventivo}
				\label{fig:preventivo}
			\end{figure}
			
		\subsubsection{Prestazione}
			
			Gli attributi dell'entità \emph{Prestazione} vengono esplicitati dal diagramma in figura \ref{fig:prestazione}.
			
			\emph{Prestazione} è composta dagli attributi "Preventivo" (codice identificativo del preventivo di riferimento), "Tempi Esecuzione" (giorni necessari effettivamente all'esecuzione del lavoro preventivato), "Malfunzionamento" (descrizione breve della natura e dell'origine del malfunzionamento riscontrato), "Procedimento" (descrizione concisa ed essenziale del procedimento utilizzato per eliminare i malfunzionamenti), "Costo Servizi" (attributo composto dal costo \emph{effettivo} dei servizi aggiuntivi e della manodopera).
		
			Dovendo tener traccia del preventivo di riferimento a fronte di una prestazione fornita, abbiamo scelto l'attributo \emph{Preventivo} come chiave primaria, dal momento che non vi possono essere più prestazioni a fronte dello stesso preventivo.
		
			\begin{figure}[H]
				\centering
				\includegraphics[width=9cm]{images/finitures/prestazione.png}
				\caption{Sviluppo dell'entità Prestazione}
				\label{fig:prestazione}
			\end{figure}
		
		\subsubsection{Componente}
			
			Nel diagramma in figura \ref{fig:componente} troviamo l'entità \emph{Componente} ed i relativi attributi.
			
			L'entità \emph{Componente} comprende gli attributi "Codice" (identificativo numerico interno del componente), "Nome", "Validità" (giorni dalla data di acquisto dopo i quali il componente diventa inutilizzabile), "Quantità Minima" (quantitativo minimo da avere sempre in magazzino), "Prezzo Vendita" (prezzo unitario al quale il componente viene venduto). 
			
			\begin{figure}[H]
				\centering
				\includegraphics[width=9cm]{images/finitures/componente.png}
				\caption{Sviluppo di Componente}
				\label{fig:componente}
			\end{figure}
		
		\subsubsection{Fattura}
		
			Nel diagramma in figura \ref{fig:fattura}, l'entità \emph{Fattura} ed i suoi attributi.
		
			\begin{figure}[H]
				\centering
				\includegraphics[width=9cm]{images/finitures/fattura.png}
				\caption{Sviluppo di Transazione}
				\label{fig:fattura}
			\end{figure}
			
			\emph{Fattura} è composta dagli attributi "Numero Progressivo" e "Anno" (coppia di attributi identificatori, derivano direttamente dalla struttura reale delle fatture), "Totale" (descrive il prezzo totale della prestazione, è composto da "Imposte", "Imponibile", "Sconto"\footnote{Quantità espressa in percentuale. Consultare \ref{rv:sconto}. }, "Incentivi"), "Data Emissione", "Sistema Pagamento" (specifica uno dei due sistemi di pagamento accettati: rimessa diretta e rimessa differita), "Tipo Pagamento" (metodologie di pagamento accettate: bonifico, contanti o assegno), "Stato Pagamento" (attributo booleano che permette di distinguere le fatture saldate da quelle non ancora pagate), "Data Scadenza" (data entro la quale la fattura deve essere saldata).
		
		\subsubsection{Transazione}
			
			Il diagramma in figura \ref{fig:transazione} raffigura lo sviluppo degli attributi dell'entità \emph{Transazione}.
			
			L'entità \emph{Transazione} è semplicemente composta dagli attributi "Codice", "Quota" (ammontare della transazione di denaro: quantità positiva per le transazioni entranti, negativa per quelle uscenti), "Data".
						
			\begin{figure}[H]
				\centering
				\includegraphics[width=4.5cm]{images/finitures/transazione.png}
				\caption{Sviluppo di Transazione}
				\label{fig:transazione}
			\end{figure}
		
		\subsubsection{Raffinamenti Successivi}
			
			Esplicitati gli attributi delle principali entità, procediamo a legarle tra loro sviluppando le relationship ed alcune entità minori.
			
			Ripercorrendo i passi dello sviluppo dello scheletro del diagramma ER, partiamo dalle relationship che legano le entità \emph{Cliente}, \emph{Autovettura}, \emph{Preventivo} (diagramma in figura \ref{fig:cliente_autovettura_preventivo}).		
		
			\begin{figure}
				\centering
				\includegraphics[width=13cm]{images/finitures/cliente_autovettura_preventivo.png}
				\caption{Sviluppo delle relationship che legano Cliente, Autovettura e Preventivo}
				\label{fig:cliente_autovettura_preventivo}
			\end{figure}
			
			Chiaramente ad ogni cliente registrato, saranno associate una o più autovetture di sua propiertà. Ad ogni autovettura saranno associati uno o più preventivi di interventi riferiti all'autovettura stessa (Diagramma in figura \ref{fig:cliente_autovettura_preventivo}).
			
			Se l'intervento preventivato viene realizzato, al preventivo sarà associata una ed una sola prestazione. La stipulazione del preventivo non è vincolante nei confronti del cliente, quindi non è vero che ad ogni preventivo corrisponde una prestazione (Diagramma in figura \ref{fig:preventivo_prestazione}).
			
			\begin{figure}[H]
				\centering
				\includegraphics[width=13cm]{images/finitures/preventivo_prestazione.png}
				\caption{Sviluppo della relationship che lega Preventivo e Prestazione}
				\label{fig:preventivo_prestazione}
			\end{figure}
			
			I componenti previsti nelle riparazioni vengono descritti tramite la relazione \emph{Previsione} che lega le entità \emph{Componente} e \emph{Preventivo}. In ogni preventivo si può prevedere di utilizzare nessuno, uno o più componenti. L'utilizzo di uno stesso componente può essere previsto - ovviamente - nella formulazione di più preventivi.
			Si consulti il diagramma in figura \ref{fig:preventivo_componente}.
			
			\begin{figure}[H]
				\centering
				\includegraphics[width=13cm]{images/finitures/preventivo_componente.png}
				\caption{Sviluppo di Preventivo e Componente}
				\label{fig:preventivo_componente}
			\end{figure}
			
			L'attributo "Ubicazione" della relazione \emph{Previsione} rappresenta l'ubicazione dei componenti utilizzati nelle installazioni di nuovi impianti (si consultino anche le specifiche riguardanti i preventivi alla sottosezione "Frasi relative ai Preventivi" \ref{sec:frasi_preventivi}).
			L'attributo "Prezzo Unitario" della relazione \emph{Previsione} si rivela necessario, in quanto il prezzo di vendita dei singoli componenti è soggetto a variazioni nel tempo.
			
			Per la registrazione degli articoli acquistati sono state introdotte in fase di sviluppo dello scheletro dello schema ER le entità \emph{Ordine} e \emph{Fornitura}. Tali entità, prese singolarmente, sono poco significative, essendo fortemente legate tra di loro (si faccia riferimento al diagramma in figura \ref{fig:fornitore_ordine_fornitura_componente}).
			
			I contratti di acquisto con i fornitori vengono modellati dall'entità \emph{Ordine}. Naturalmente presso lo stesso fornitore si possono effettuare più ordini, ma un ordine si riferisce ad un singolo fornitore. \emph{Ordine} e \emph{Fornitore} sono legati dalla relationship \emph{Acquisto}.
			
			\begin{figure}[H]
				\centering
				\includegraphics[width=13cm]{images/finitures/fornitore_ordine_fornitura_componente}
				\caption{Sviluppo delle relazioni che legano il Fornitore, l'Ordine d'acquisto, le Forniture e i Componenti}
				\label{fig:fornitore_ordine_fornitura_componente}
			\end{figure}
			
			Ogni ordine è composto da una o più forniture, le quali, a loro volta, sono composte da uno o più articoli dello stesso componente. Ad ogni istanza dell'entità \emph{Fornitura} si associa - tramite la relationship \emph{Riferimento} - una ed una sola istanza dell'entità \emph{Componente}. Di contro, lo stesso componente può essere acquistato in diverse forniture.
			Quindi ogni istanza di \emph{Fornitura} sarà associata ad una ed una sola istanza di \emph{Ordine} tramite la relationship \emph{Formazione}. Un ordine vedrà associate a sè una o più forniture.
			
			\begin{figure}
				\centering
				\includegraphics[width=13cm]{images/finitures/componente_fornitura_ordine_magazzino.png}
				\caption{Introduzione del Magazzino}
				\label{fig:componente_fornitura_ordine_magazzino}
			\end{figure}
			
			Un'ulteriore entità da aggiungere a questo gruppo è quella del magazzino. Abbiamo definito \emph{Magazzino} come una raccolta di forniture attive, cioè di forniture i quali articoli sono ancora presenti nel magazzino fisico, pronti per essere utilizzati. Legando \emph{Magazzino} con \emph{Fornitura} si modella tale associazione. La relationship \emph{Composizione} associa ad ogni istanza di \emph{Magazzino} una ed una sola istanza di \emph{Fornitura}. Da notare che il codice della fornitura e il codice del componente di riferimento, formano la chiave primaria di tale entità.
			Fare riferimento al diagramma in figura \ref{fig:componente_fornitura_ordine_magazzino}.
			
			Da notare che l'attributo "Quantità" dell'entità \emph{Magazzino} descrive il numero di articoli di uno specifico componente, acquistati in una certa fornitura, ancora disponibili in magazzino.
			
			All'esecuzione di una prestazione, come da specifiche, è necessario specificare il tipo e la quantità di articoli utilizzati. La prima soluzione che ci è sembrata valida è stata quella di associare, ad ogni istanza dell'entità \emph{Prestazione} le istanze interessate dell'entità \emph{Componente} attraverso la relationship \emph{Utilizzo}. Tale relationship avrebbe avuto l'attributo "Quantità", necessario per specificare la quantità degli articoli utilizzati per ogni componente.
			
			Tuttavia, tale design si è rivelato non adeguato a soddisfare tutte le specifiche. Il problema più evidente risiedeva nel fatto che, essendo le istanze dell'entità \emph{Componente} composte da informazioni descrittive, pressocchè invarianti (eccezion fatta per quanto riguarda il "Prezzo di Vendita"), non vi è il modo per risalire al preciso articolo fisico utilizzato nella riparazione\footnote{Non potendo risalire al preciso articolo utilizzato, non si ha a disposizione la data di acquisto, quindi viene meno la realizzabilità del meccanismo che permette di utilizzare per primi gli articoli dei componenti la cui data di scadenza è più vicina di altri.}.
			
			All'entità \emph{Prestazione} vengono quindi associate zero, una o più istanze dell'entità \emph{Fornitura}, avendo così a disposizione sia le informazioni che descrivono genericamente il componente, sia quelle che caratterizzano con precisione l'articolo utilizzato nella prestazione. 

			\begin{figure}
				\centering
				\includegraphics[width=11.5cm]{images/finitures/prestazione_fornitura.png}
				\caption{Utilizzo di componenti un una prestazione}
				\label{fig:ordine_fornitore}
			\end{figure}
			
			L'attributo "Quantità" della relationship \emph{Utilizzo} non necessita di ulteriori spiegazioni, mentre l'attributo "Prezzo Unitario" si rende necessario, in quanto il prezzo di vendita di un componente, ragionevolmente, varia nel tempo.
			
			Ad esecuzione ultimata di una prestazione avviene la registrazione della fattura. Ad ogni istanza dell'entità \emph{Prestazione} sarà associata, tramite la relationship \emph{Registrazione} obbligatoriamente una ed una sola istanza dell'entità \emph{Fattura}. 		
			
			Le istanze dell'entità \emph{Fattura} vengono identificate dalla coppia di attributi "Numero Progressivo" ed "Anno", così come avviene nella realtà di interesse\footnote{Le fatture vengono identificate dall'anno di emissione e dal numero progressivo. Ogni anno tale numero viene azzerato.}.
			
			\begin{figure}
				\centering
				\includegraphics[width=11.5cm]{images/finitures/prestazione_fattura.png}
				\caption{Sviluppo della relationship tra Prestazione e Fattura}
				\label{fig:prestazione_fattura}
			\end{figure}
			
			Gli sviluppi dei diagrammi introdotti fin'ora permettono di affrontare il legame di \emph{Transazione} con le altre entità. A quest'ultima si possono associare istanze di tutte le entità che modellano dati di porzioni di processo che prevedono il verificarsi di transazioni monetarie. Alla stipulazione di un preventivo può essere richiesto il versamento di un acconto, alla consegna gli ordine sarà necessario effettuare una versamento al fornitore, mensilmente bisognerà registrare gli stipendi versati agli operatori e quando una fattura viene saldata bisognerà registrare tale transazione di denaro.
			
			\begin{figure}
				\centering
				\includegraphics[width=13cm]{images/finitures/transazione_fattura_preventivo_operatore_ordine.png}
				\caption{Sviluppo delle relationship con cui Transazione si lega alle altre entità}
				\label{fig:transazione_fattura_preventivo_operatore_ordine}
			\end{figure}
			
			Nel diagramma in figura \ref{fig:transazione_fattura_preventivo_operatore_ordine} vi è la rappresentazione di come le entità \emph{Preventivo}, \emph{Ordine}, \emph{Operatore}, \emph{Fattura} vengono associate a \emph{Transazione}.
			
			Esaminiamo le ultime componenti del diagramma ER che non sono state ancora analizzate.
			
			In figura \ref{fig:operatore_prestazione} il diagramma descrive la relationship \emph{Occupazione} che associa le istanze di \emph{Prestazione} a quelle di \emph{Operatore}. Ad ogni istanza di \emph{Prestazione} infatti devono essere associate una o più instanze di \emph{Operatore}, in modo da tener traccia dei dipendenti che sono stati impiegati nell'esecuzione della prestazione ad un'autovettura. Ovviamente la stessa istanza di \emph{Operatore} può essere associata a più istanze di \emph{Prestazione}.
			
			\begin{figure}
				\centering
				\includegraphics[width=11.5cm]{images/finitures/operatore_prestazione.png}
				\caption{Sviluppo della relationship tra Prestazione e Operatore}
				\label{fig:operatore_prestazione}
			\end{figure}
			
			Riguardo gli operatori è necessario, come da specifiche, registrarne le presenze e gli orari di lavoro. L'entità \emph{Turno}, legata ad \emph{Operatore} tramite la relationship \emph{Presenza} (diagramma in figura \ref{fig:operatore_turno}), assolve tale funzione.
			
			\begin{figure}
				\centering
				\includegraphics[width=11.5cm]{images/finitures/operatore_turno.png}
				\caption{Turni degli Operatori}
				\label{fig:operatore_turno}
			\end{figure}
			
			L'ultimo punto da sviluppare consiste nella gestione dei recapiti, di qualunque natura essi siano. L'entità \emph{Recapito} è costituita dagli attributi "Recapito" (che è anche chiave primaria) e dall'attributo "Tipo" (consultare le Regole Aziendali alla sezione \ref{sec:business_rules} per i valori che tale attributo può assumere).
			Ad ogni istanza di \emph{Persona} devono essere associati una o più istanze di \emph{Recapito}.
			
			\begin{figure}[H]
				\centering
				\includegraphics[width=11.5cm]{images/finitures/persona_rubrica.png}
				\caption{Recapiti associati ad una Persona}
				\label{fig:persona_recapito}
			\end{figure}
	
	% Diagramma definitivo
	\subsection{Diagramma Entity-Relationship}
		
		L'intero diagramma ER si può trovare in figura \ref{fig:er}.
			
		\begin{sidewaysfigure}
			\centering
			\includegraphics[width=22cm]{images/finitures/schema.png}
			\caption{Scheletro del diagramma ER}
			\label{fig:er}
		\end{sidewaysfigure}

		\newpage

	\subsection{Analisi Qualitativa dello Schema ER}
		
		Effettuiamo una breve analisi in termini qualitativi del diagramma ER sviluppato.
		
		\begin{description}
			\item[Correttezza] Il diagramma sviluppato fa un uso sintatticamente e semanticamente corretto dei costrutti disponibili del modello Entità-Relazione.
			\item[Completezza] Confrontando il diagramma risultante con le specifiche che abbiamo individuato nell'Analisi dei Requisiti (sezione \ref{sec:req_analysis}), reputiamo che quest'ultime siano soddisfatte.
			\item[Leggibilità] Abbiamo strutturato graficamente il diagramma in modo da favorirne il più possibile la leggibilità. In particolare ci siamo focalizzati sul minimizzare il numero di intersezioni tra gli archi che collegano entità e relationship, non riuscendo tuttavia ad evitarle del tutto.
			\item[Minimalità] Il diagramma non è del tutto privo di parti ridondanti. Gli attributi \emph{Imponibile} ed \emph{Imposte} relative all'entità \emph{Fattura} introductono ridondanza nella rappresentazione dell'imformazione (consultare per maggiori informazioni le Regole di Derivazione \ref{rd:imponibile} e \ref{rd:imposte}). Valuteremo nelle prossime fasi progettuali se eliminare o meno tale ridondanza.
		\end{description}
		
		Il diagramma, nonostante l'assenza di minimalità, risulta valido, adeguato per procedere ai successivi passi progettuali.
	
	% Sezione documentativa della progettazione concettuale
	\subsection{Dizionario dei Dati}
	\label{sec:data_dict}
		
		\subsubsection{Entità}
		\label{sec:entities}
			
			\begin{description}
				\item[NB] Esplicitiamo gli attributi composti elencando tra le parentesi quadre gli attributi semplici di cui sono costituiti.
			\end{description}
	
			{\small
			\begin{longtable}{| p{2cm} | p{4cm} | p{4cm} | p{2cm} |}
				
				\hline
				\textbf{Nome} & 
				\textbf{Descrizione} & 
				\textbf{Attributi} & 
				\textbf{Identificatore} \\ 
				\hline
				
				\endfirsthead
				
				\hline
				\textbf{Nome} & 
				\textbf{Descrizione} & 
				\textbf{Attributi} & 
				\textbf{Identificatore} \\ 
				\hline
				
				\endhead
				
				Persona &
				Soggetto generico che intrattenga rapporti di ogni tipo con l'azienda. &
				Codice Fiscale o P.IVA (Stringa), Indirizzo [Città (Stringa), Via (Stringa), Civico (Numerico), CAP (Numerico)] &
				Codice Fiscale o P.IVA (Stringa)
				\\ \hline

				Cliente &
				Soggetto che necessita di un servizio da parte dell'azienda. &
				// &
				//
				\\ \hline

				Privato &
				Cliente non dotato di partita IVA. &
				Attributi di Persona, Nome (Stringa), Cognome (Stringa), Numero Documento Identità (Stringa) &
				//
				\\ \hline

				Azienda &
				Cliente dotato di partita IVA. &
				Attributi di Persona, Ragione Sociale (Stringa) &
				//
				\\ \hline

				Fornitore &
				Azienda che abbia fornito all’officina qualsiasi tipo di componente necessario. &
				Attributi di Persona, Ragione Sociale (Stringa), Tempi Consegna (Numerico), Modalità Pagamento (Stringa), IBAN (Stringa) &
				//
				\\ \hline

				Operatore &
				Lavoratore dell'azienda. &
				Attributi di Persona, Nome (Stringa), Cognome (Stringa), IBAN (Stringa), Stipendio (Numerico), Retribuzione Oraria (Numerico), Modalità Riscossione (Stringa), Dati Anagrafici [Comune Nascita (Stringa), Provincia (Stringa), Data Nascita (Data)] &
				//
				\\ \hline

				Autovettura &
				Automobile di un cliente che debba subire o abbia già subito un intervento da parte dei lavoratori dell’azienda. &
				Targa (Stringa), Modello (Stringa), Telaio (Stringa), Cilindrata (Numerico), Anno Immatricolazione (Numerico), Ultima Revisione (Data), Ultimo Collaudo (Data), Marca (Stringa) &
				Targa (Stringa)
				\\ \hline

				Preventivo &
				Stima dei costi, dei tempi e dei componenti necessari relativi all’esecuzione di un intervento su un’autovettura. &
				Codice (Numerico), Data Emissione (Data), Data Inizio (Data), Categoria (Stringa), Costo Servizi [Servizi Aggiuntivi (Numerico), Manodopera (Numerico)], Tempo Stimato (Numerico), Sintomi (Stringa), Sistema Alimentazione (Stringa) &
				Codice (Numerico)
				\\ \hline

				Componente &
				Qualsiasi oggetto fisico necessario alla corretta esecuzione di una riparazione o di una installazione di un impianto su di un’autovettura. &
				Codice (Numerico), Quantità Minima (Numerico), Prezzo Vendita (Numerico), Validità (Numerico), Nome (Stringa) &
				Codice (Numerico)
				\\ \hline

				Fornitura &
				Insieme dello stesso componente inviata da un fornitore. &
				Codice (Numerico), Prezzo Unitario (Numerico), Quantità (Numerico) &
				Codice (Numerico)
				\\ \hline

				Ordine &
				Insieme di forniture inviate nello stesso momento e dallo stesso ordine. &
				Codice (Numerico), Data Consegna (Data), Data Emissione (Data) &
				Codice (Numerico)
				\\ \hline
				
				Magazzino &
				Insieme di tutte le forniture non esaurite. &
				Codice Componente (Numerico), Quantità (Numerico) &
				Codice Componente (Numerico), Codice (di \emph{Fornitura})
				\\ \hline

				Prestazione &
				Attività eseguita dagli operatori dell’officina su di un’autovettura. &
				Codice (di \emph{Preventivo}), Tempi Esecuzione (Numerico), Data Fine (Data), Costo Servizi [Manodopera (Numerico), Servizi Aggiuntivi (Numerico)], Malfunzionamento (Stringa), Procedimento (Stringa) &
				Codice (di \emph{Preventivo})
				\\ \hline

				Turno &
				Arco temporale specifico in cui gli operatori compiono le loro mansioni. &
				Ora Inizio (Numerico), Ora fine (Numerico), Data (Data) &
				Ora Inizio (Numerico), Data (Data), Codice Fiscale o P.IVA (di \emph{Operatore})
				\\ \hline

				Transazione &
				Flusso di denaro uscente o entrante nella cassa dell’attività. &
				Codice (Numerico), Quota (Numerico), Data (Data) &
				Codice (Numerico)
				\\ \hline

				Fattura &
				Documento fiscale relativo ad un pagamento da ricevere da parte di un cliente. &
				Numero Progressivo (Numerico), Anno (Numerico), Data Emissione (Data), Totale [Imponibile (Numerico), Imposte (Numerico), Sconto (Numerico), Incentivi (Numerico)], Sistema Pagamento (Stringa), Tipo Pagamento (Stringa), Stato Pagamento (Stringa), Data Scadenza (Data) &
				Numero Progressivo (Numerico), Anno (Numerico)
				\\ \hline
				
				Recapito &
				Numero telefonico, indirizzo email o sito web. Qualsiasi recapito telematico utile a contattare una Persona. &
				Recapito (Stringa), Tipo(Stringa) &
				Recapito (Stringa)
				\\ \hline
				
			\end{longtable}
			}

		\subsubsection{Relazioni}
		\label{sec:relationships}
			{\small
			\begin{longtable}{| p{2cm} | p{4cm} | p{3cm} | p{3cm} |}
				
				\hline
				\textbf{Nome} & 
				\textbf{Descrizione} & 
				\textbf{Entità Coinvolte} & 
				\textbf{Attributi} \\ 
				\hline
				
				\endfirsthead
				
				\hline
				\textbf{Nome} & 
				\textbf{Descrizione} & 
				\textbf{Entità Coinvolte} & 
				\textbf{Attributi} \\ 
				\hline
				
				\endhead
				
				Esecuzione &
				Associa ad un Preventivo una Prestazione &
				Preventivo (0, 1), Prestazione (1, 1) &

				\\ \hline

				Previsione &
				Associa i Componenti previsti in fase di stipulazione dei Preventivi &
				Componente (0, N), Preventivo (0, N) &
				Quantità (Numerico) indica la quantità del componente che si prevede di utilizzare; Ubicazione (Stringa) indica la posizione del componente preventivato nell'autovettura; Prezzo Unitario (Numerico) indica il prezzo attuale del componente.
				\\ \hline

				Formazione &
				Associa le forniture che compongono un ordine &
				Fornitura (1, 1), Ordine (1, N) &

				\\ \hline

				Composizione &
				Associa le Forniture che compongono il Magazzino aziendale &
				Magazzino (0, N), Fornitura (0, 1) &

				\\ \hline

				Riferimento &
				Associa i Componenti che descrivono una Fornitura &
				Componente (0, N), Fornitura (1, 1) &

				\\ \hline

				Utilizzo &
				Associa i Componenti relativi ad un Fornitura effettivamente usati per compiere una prestazione &
				Fornitura (0, N), Prestazione (0, N) &
				Quantità (Numerico) indica la quantità del componente che si è utilizzata; Prezzo Unitario (Numerico) indica il prezzo di vendita del componente al momento dell'utilizzo.
				\\ \hline

				Occupazione &
				Associa un Operatore ad una Prestazione da svolgere &
				Operatore (1, N), Prestazione (1, N) &
				\\ \hline

				Presenza &
				Associa un Operatore con il Turno di lavoro effettuato &
				Operatore (1, N), Turno (1, 1) &
				\\ \hline

				Possesso &
				Associa una o più Autovetture ad un Cliente &
				Cliente (1, N), Autovettura (1, 1) &
				\\ \hline

				Richiesta &
				Associa un Preventivo riferito ad una Prestazione da richiedere su una determinata Autovettura &
				Autovettura (1, N), Preventivo (1, 1) &
				\\ \hline

				Acquisto &
				Associa un Ordine effettuato da un determinato Fornitore &
				Ordine (1, 1), Fornitore (1, N) &
				\\ \hline

				Rubrica &
				Associa una generica Persona ai sui Recapiti &
				Persona (1, N), Recapito (1, N) &
				\\ \hline

				Acconto &
				Associa un Preventivo e la Transazione monetaria che un cliente può lasciare &
				Preventivo (0, 1), Transazione (0, 1) &
				\\ \hline

				Registrazione &
				Associa una Prestazione ad una Fattura &
				Prestazione (1, 1), Fattura (1, 1) &
				\\ \hline

				Pagamento &
				Associa il manifestarsi della Transazione di pagamento riferita ad una Fattura &
				Fattura (0, 1), Transazione (0, 1) &
				\\ \hline

				Versamento &
				Associa la Transazione monetaria relativa ad un Ordine &
				Ordine (0, 1), Transazione (0, 1) &
				\\ \hline

				Stipendio &
				Associa la Transazione relativa al pagamento dello stipendio di un Operatore &
				Operatore (0, N), Transazione (0, 1) &
				\\ \hline

			\end{longtable}
			}
		
	
	\subsection{Regole Aziendali}
	\label{sec:business_rules}
	
		\subsubsection{Regole di Vincolo}
		
			\setenumerate[1]{label=RV\arabic*}
			\begin{enumerate}
				
				% Si possono implementare le reference agli elementi delle liste
				% impostando come sempre \label{itm:nome} e richiamandola con 
				% \ref{itm:nome}. Si visualizza tutta la stringa che enumera
				% l'elemento della lista
				
				% Persona
				\item \emph{Codice Fiscale o P.IVA} relativo all'entità \emph{Persona} deve essere o una stringa alfanumerica di 16 caratteri, nel caso in cui rappresenti il codice fiscale di un privato, o una stringa numerica di 11 caratteri, nel caso in cui rappresenti la partita iva di un soggetto fiscale.
				\item \emph{CAP} relativo all'entità \emph{Persona} deve essere una stringa numerica di 5 caratteri.
				
				% Recapito
				\item \emph{Tipo} relativo all’entità \emph{Recapito} deve essere uno tra i seguenti: "telefono", "fax", "tel\_fax", "sito\_web", "email".
				
				% Privato
				\item \emph{Numero Documento Identità} relativo all’entità \emph{Privato} deve essere una stringa alfanumerica di 9 caratteri.
				
				% Operatore
				\item \emph{Provincia} relativa all'entità \emph{Operatore} deve essere una stringa alfabetica di 2 caratteri maiuscoli.
				\item \emph{Stipendio} relativo all'entità \emph{Operatore} deve essere un numero maggiore di zero o NULL.
				\item \emph{Retribuzione Oraria} relativo all'entità \emph{Operatore}  deve essere un numero maggiore di zero o NULL.
				\item \emph{Stipendio} e \emph{Retribuzione Oraria} relativi all'entità \emph{Operatore} non possono essere entrambi NULL, nè entrambi diversi da NULL.
				\item \emph{Modalità Riscossione} relativo all'entità \emph{Operatore} deve essere una tra le seguenti: "bonifico", "assegno", "contanti". Se \emph{Stipendio}, relativo alla stessa entità, è maggiore o uguale a 1000, allora \emph{Modalità Riscossione} non può essere "contanti".
				\item \emph{IBAN} relativo all’entità \emph{Operatore} deve essere una stringa alfanumerica di 27 caratteri. Se \emph{Modalità Riscossione} è "bonifico", allora non può essere NULL.
				
				% Turno
				\item \emph{Ora Inizio} e \emph{Ora Fine} relativi all’entità \emph{Turno} devono essere orari. \emph{Ora Inizio} deve essere antecedente a \emph{Ora Fine}.
				
				% Fornitore
				\item \emph{Modalità Pagamento} relativo all'entità \emph{Fornitore} deve essere una tra le seguenti: "bonifico", "assegno".
				\item \emph{IBAN} relativo all’entità \emph{Fornitore} deve essere una stringa alfanumerica di 27 caratteri. Se \emph{Modalità Pagamento} è "bonifico" allora non può essere NULL.
				
				% Autovettura
				\item \emph{Targa} relativo all'entità \emph{Autovettura} deve essere una stringa alfanumerica di 8 caratteri se \emph{Anno Immatricolazione} è maggiore di 1927 e minore di 1994, deve essere una stringa alfanumerica di 7 caratteri se \emph{Anno Immatricolazione} è maggiore di 1994, deve essere una stringa alfanumerica di 7 o di 8 caratteri se \emph{Anno Immatricolazione} è uguale a 1994.
				\item \emph{Telaio} relativo all'entità \emph{Autovettura} deve essere una stringa alfanumerica di 17 caratteri.
				
				% Preventivo
				\item \emph{Categoria} relativa all'entità \emph{Preventivo} deve essere una tra le seguenti: "riparazione", "installazione\_impianto\_metano", "instalazione\_impianto\_gpl", "collaudo", "revisione".
				\item \emph{Sistema Alimentazione} relativo all'entità \emph{Preventivo} deve essere NULL se \emph{Categoria}, relativa alla stessa entità, non è "installazione\_impianto\_gpl" o "installazione\_impianto\_metano", altrimenti deve essere una tra le seguenti: "aspirazione", "iniezione".
				
				% Componente				
				\item \emph{Validità} relativo all’entità \emph{Componente} deve essere un numero maggiore di zero se il componente ha una data di scadenza, uguale a zero altrimenti.
				\item \emph{Quantità Minima} relativa all’entità \emph{Componente} deve essere un numero maggiore di zero se il componente prevede una quantità minima, uguale a zero altrimenti.
				
				% Previsione
				\item \emph{Ubicazione} relativo alla relationship \emph{Previsione} deve essere NULL se \emph{Categoria}, relativa all'entità \emph{Preventivo}, non è "installazione\_impianto\_metano" o "installazione\_impianto\_gpl", altrimenti deve essere una tra le seguenti: "motore", "bagagliaio".
				\item \emph{Quantità} relativa alla relationship \emph{Previsione} è un numero e deve essere maggiore di 0.
				
				% Prestazione
				\item \emph{Data Fine} relativa all'entità \emph{Prestazione} deve essere successiva a \emph{Data Emissione}, relativa all'entità \emph{Preventivo}.
				
				% Utilizzo
				\item \emph{Quantità} relativa alla relationship \emph{Utilizzo} è un numero e deve essere maggiore di 0.
				
				% Fornitura		
				\item \emph{Quantità} relativa all'entità \emph{Fornitura} deve essere un numero maggiore di zero.
				\item \emph{Prezzo Unitario} relativa all'entità \emph{Fornitura} deve essere un numero maggiore di zero.
				
				% Ordine
				\item \emph{Data Emissione} e \emph{Data Consegna} relativi all'entità \emph{Ordine} sono due date e \emph{Data Consegna} non può essere antecedente a \emph{Data Emissione}.
				
				% Fattura
				\item \emph{Tipo Pagamento} relativo all'entità \emph{Fattura} deve essere una tra le seguenti: "bonifico", "assegno", "contanti".
				\item \emph{Sistema Pagamento} relativo all'entità \emph{Fattura} deve essere una tra le seguenti: "rimessa\_diretta", "rimessa\_differita".
				\item \emph{Data Emissione} e \emph{Data Scadenza} relativi all'entità \emph{Fattura} sono date e \emph{Data Scadenza} non può essere antecedente a \emph{Data Emissione}. Se \emph{Sistema Pagamento} relativo alla stessa entità è "rimessa\_diretta", allora \emph{Data Scadenza} deve essere uguale a \emph{Data Emissione}.
				\item \emph{Stato Pagamento} relativo all’entità \emph{Fattura} deve essere un valore booleano non NULL. Assume "TRUE" se la fattura è stata saldata, "FALSE" altrimenti.
				\item \label{rv:sconto} \emph{Sconto} relativo all'entità \emph{Fattura} deve essere un numero decimale maggiore o uguale a 0 e minore di 100.
				
				% Transazione
				\item \emph{Quota} relativa all'entità \emph{Transazione}, relativamente ad un'istanza dell'entità stessa associata ad un'istanza di \emph{Fattura}, è un numero pari alla somma dei valori degli attributi (relativi a \emph{Fattura}) \emph{Imponibile} ed \emph{Imposte}, meno il valore di \emph{Incentivi}.
				\item \emph{Quota} relativa all'entità \emph{Transazione}, relativamente ad un'istanza dell'entità stessa associata ad un'istanza di \emph{Preventivo}, se attributo di un'istanza associata ad un'istanza di \emph{Preventivo}, è un numero maggiore di zero e minore del 70\% del Costo Stimato del Preventivo.
				\item \emph{Quota} relativa all'entità \emph{Transazione}, relativamente ad un'istanza dell'entità stessa associata ad un'istanza di \emph{Operatore}, è un numero pari all'attributo \emph{Stipendio} di \emph{Operatore}, oppure è pari al numero delle ore di lavoro effettuate, moltiplicate per \emph{Retribuzione Oraria} di \emph{Operatore}.
				\item \emph{Quota} relativa all'entità \emph{Transazione}, relativamente ad un'istanza dell'entità stessa associata ad un'istanza di \emph{Ordine}, è un numero pari alla somma dei valori ottenuti moltiplicando \emph{Prezzo Unitario} e \emph{Quantità} relativi alle istanze di \emph{Fornitura} associate all'istanza interessata di \emph{Ordine}.

			\end{enumerate}
		
		\subsubsection{Regole di Derivazione}
		
			\setenumerate[1]{label=RD\arabic*}
			\begin{enumerate}

				\item \label{rd:imponibile} \emph{Imponibile} relativo all'entità \emph{Fattura} è la somma dei valori ottenuti moltiplicando l'attributo \emph{Prezzo Unitario} di ogni istanza associata alla prestazione di riferimento, per \emph{Quantità} (entrambi relativi alla relationship \emph{Utilizzo}). A tale valore va aggiunto l'ammontare di \emph{Costo Servizi}.

				\item \label{rd:imposte} \emph{Imposte} relativo all'entità \emph{Fattura} è pari al valore dell'IVA calcolato sul valore dell'attributo \emph{Imponibile}, relativo all'entità stessa (Cfr. \ref{rd:imponibile})

				\item \label{rd:prezzo_unitario} \emph{Prezzo Unitario} relativo alla relationship \emph{Utilizzo} è pari al valore di \emph{Prezzo di Vendita} dell'entità \emph{Componente} al momento dell'inserimento
				
			\end{enumerate}

\newpage

% Sezione sulla progettazione logica
% Sezione con gli screenshot delle chiamate alla base di dati

\end{document}