\section{Analisi dei Requisiti}
	\subsection{Raccolta delle Informazioni}
	Nessun elemento del team conosceva, neppur superficialmente, la realtà d'interesse che saremmo andati a modellare, per cui l'attività di raccolta delle informazioni è risultata essere fondamentale. Per raggiungere tale obbiettivo abbiamo intervistato il titolare dell'officina e siamo riusciti a procurarci alcuni documenti.
	
		\subsubsection{Intervista}
		Abbiamo intervistato il \emph{Sig. Adriano Staffonali}, titolare di un'officina meccanica nel comune di Treia (MC). L'intervista risale al 26 Ottobre 2014. Riportiamo, qui di seguito, i passaggi fondamentali.

		% Intervista		
		\begin{description}
		 	\marginpar{Rivedere ed inserire qualche domanda finale}
			\item[A] 
				Di cosa si occupa la sua attività?
			\item[AS] 
				\emph{La mia attività è un'officina meccanica. Mi occupo di effettuare piccole e medie riparazioni di tipo meccanico ad autovetture e sono specializzato nella sostituzione, riparazione e manutenzione dei componenti elettronici. Inoltre la mia officina è autorizzata all'installazione di impianti a metano e GPL "Landi Renzo", azienda leader nel settore al livello nazionale.}
			\item[A]
				Quante persone vi lavorano?
			\item[AS]
				\emph{Attualmente solo io, ma in passato ho avuto un paio di dipendenti.}
			\item[A]
				Come si articola una tipica giornata di lavoro?
			\item[AS]
				\emph{Solitamente ho sempre degli impianti da installare, che occupano la maggior parte della giornata. Ho un calendario dove segno tutte le scadenze a cui devo tener fede. Quando arriva un cliente, che abbia bisogno di una riparazione all'auto o dell'installazione di un impianto, devo fornirgli un preventivo. Se accetta, controllo quali pezzi devo acquistare, rintraccio i fornitori e li ordino.}
			\item[A]
				Che tipo di clienti sono i suoi? Privati? Aziende? Come tiene traccia dei loro dati?
			\item[AS]
				\emph{Per lo più i miei clienti sono privati, ma mi capita di lavorare con aziende e - occasionalmente - anche con enti pubblici. Tengo traccia solamente dei clienti quando effettuano nuovi impianti, in quanto la} Landi Renzo \emph{richiede per ogni nuovo cliente una scheda d'installazione da compilare on-line contenente dati anagrafici, recapiti e dati autovettura.}
			\item[A]
				Ammesso di avere individuato il guasto e di aver ben presente quali sono i pezzi da sostituire, solitamente, quanto sono precisi i preventivi per una riparazione? E quelli per l'installazione di un impianto a metano?
 			\item[AS]
 				\emph{Per quanto riguarda le riparazioni, non si può dare sempre un preventivo preciso. Bisogna tener conto di alcuni aspetti: l'uso di pezzi di ricambio originali o meno e le ore di lavoro necessarie per effettuare la riparazione (di cui è sempre difficile effettuare previsioni precise). Per quanto riguarda l'installazione di impianti, invece, l'azienda che li produce e me li fornisce, predispone un listino prezzi completo che mi permette di effettuare preventivi in modo veloce e accurato.}
 			\item[A]
 				Non tiene uno storico delle riparazioni effettuate al fine di riutilizzare i dati per trovare soluzioni più velocemente in futuro?
 			\item[AS]
 				\emph{Uno storico no. Ho alcuni schemi tecnici che mi aiutano a risolvere il problema più velocemente. Però uno storico sarebbe utile.}
 			\item[A]
 				Per quanto riguarda i pagamenti da parte dei clienti, come si è organizzato? Inoltre, permette pagamenti dilazionati o rateizzati da parte dei clienti, che essi siano privati od aziende?
 			\item[AS]
 				\emph{Al momento utilizzo un archivio cartaceo per quanto riguarda fatture e ricevute. Pagamenti dilazionati? Raramente. Permetto solo ad alcuni clienti abituali di lasciare un acconto iniziale, per poi completare il pagamento in seguito. Ad alcune aziende, con le quali intrattengo rapporti frequentemente, permetto di effettuare pagamenti dilazionati. Quando si tratta invece di enti pubblici (ho avuto in passato rapporti commerciali con il comune di Treia) il pagamento dilazionato è l'unica soluzione.}
 			\item[A]
 				E per quanto riguarda i suoi fornitori? Le permettono pagamenti dilazionati?
 			\item[AS]
 				\emph{A dire il vero, raramente. Essendo la mia una piccola azienda, solo alcuni fornitori con cui ho instaurato un rapporto di fiducia nel tempo, mi permettono pagamenti dilazionati.}
 			\item[A]
 				Quindi lei si occupa da solo anche di tutta la contabilità, giusto?
 			\item[AS]
 				\emph{Non del tutto. Ho un commercialista. Lui si occupa di stilare il Bilancio e lo Stato Patrimoniale.}
 			\item[A]
 				Lei è solito tenere in magazzino pezzi per alcune riparazioni frequenti?
 			\item[AS]
 				\emph{Sì, cerco di avere sempre disponibili i pezzi fondamentali.}
 			\item[A]
 				Riesce a gestire adeguatamente il magazzino? Le è mai capitato di avere avuto dei prodotti che, soggetti magari all'usura del tempo, si siano rovinati?
 			\item[AS]
 				\emph{Ci sono alcuni prodotti che sono più soggetti di altri all'usura del tempo, altri invece che diventano obsoleti. Faccio un inventario completo delle rimanenze in magazzino una volta all'anno ed è un'attività che porta via molto tempo. Inoltre, quando utilizzo un pezzo per una riparazione, non vado ad aggiornare l'inventario, quindi non riesco a sapere ogni volta con precisione lo stato del magazzino.}
 			\item[A]
 				Per quanto riguarda i dati dei fornitori come ne tiene traccia? È sempre in grado di ritrovarli facilmente e immediatamente?
 			\item[AS]
 				\emph{Sinceramente no, ne tiene traccia il commercialista. Io ho solamente una rubrica cartacea con i numeri di telefono. Infatti, quando ho bisogno di ritrovarli devo ricontattarlo, a meno che non necessiti dei numeri di telefono. Un'altra eccezione sono i fornitori che inviano le loro fatture via e-mail, in quanto contenenti i dati dell'azienda di riferimento, è ovvio però che non sia il metodo migliore.}
 			\item[A]
 				Lei lavora da solo, ma ha detto di aver avuto un dipendente. Che tipo di contratto aveva? Si occupava personalmente delle buste paga?
 			\item[AS]
 				\emph{Giornalmente segnavo le sue ore di lavoro, quindi gli versavo l'importo a fine mese. Riuscivo ad occuparmene tranquillamente, era un solo dipendente d'altronde, ma se in futuro avessi bisogno di assumere più di una persona, dovrei adottare un altro metodo.}
 			
		\end{description}
				
		\subsubsection{Documenti raccolti}
		\subsubsection{Analisi dei processi interni}
		
	\subsection{Requisiti Espressi nel Linguaggio Naturale}
		A partire dall’analisi dell’intervista e dall’analisi dei documenti in nostro possesso, abbiamo elaborato quelli che sono, a nostro avviso, i requisiti della base di dati che andremo a sviluppare. 
		
		Il nostro obbiettivo è quello di sviluppare una base di dati per la gestione di un’officina meccanica di piccole medie dimensioni specializzata nell’installazione di impianti a metano e a GPL, ma che effettua anche riparazioni di natura meccanica ed elettronica alle autovetture. È nostra intenzione sviluppare una base di dati che risulti solida e duratura, il meno possibile sensibile alle variazioni giuridiche. 
		
		Bisognerà gestire i dati riguardanti i clienti e le loro autovetture, quelli riguardanti i fornitori e i componenti presenti in magazzino, i dati dei dipendenti. Si vogliono tracciare i dati riguardanti i preventivi emessi dall’azienda e quelli delle prestazioni effettuate, in modo da agevolare la formulazione del preventivo sia in termini di tempistiche, sia in termini di precisione. Si vogliono conoscere i componenti più utilizzati nelle riparazioni e nelle installazioni, al fine di ottimizzare le rimanenze in magazzino e ridurre gli sprechi. Si vuole anche tenere traccia delle transazioni monetarie entranti (pagamenti dei clienti per le prestazioni ricevute) ed uscenti (pagamenti ai fornitori ed ai dipendenti), ma senza entrare nel merito dei contratti stipulati con i fornitori e con i dipendenti.
		
		Per quanto riguarda i clienti non dotati di partita iva, si vogliono conoscere il codice fiscale, il nome, il cognome, l’indirizzo di residenza, i vari recapiti. Per i clienti forniti di partita iva, allo stesso modo, si vuole tener traccia del codice fiscale (che corrisponde alla prima partita iva rilasciata dalla Agenzia delle Entrate), della partita iva, della ragione sociale e dell’indirizzo della sede legale. Nel caso in cui un cliente non dotato di partita iva richieda l’installazione di un nuovo impianto, sarà necessario conoscere anche il codice identificativo del documento di identità per le comunicazioni con la Motorizzazione Civile. 
		
		Per quanto riguarda le autovetture sarà necessario conoscere la targa, la marca, il nome del modello. Se per un’autovettura viene richiesta l’installazione di un impianto, sarà necessario conoscere anche la cilindrata, l’anno di immatricolazione e la data dell’ultima revisione, data dell’ultimo collaudo. 
		
		Riguardo i fornitori si vogliono conoscere la partita iva, la ragione sociale, i vari recapiti, i tempi medi di consegna. 
		
		Dei dipendenti si vuole tener traccia di codice fiscale, nome, cognome, luogo di nascita, data di nascita, indirizzo di residenza, retribuzione oraria, modalità di riscossione ed - eventualmente - il codice IBAN. Si vogliono anche conoscere le presenze che i dipendenti effettuano, tenendo conto dell’ora di inizio del turno, l’ora di fine e la data di rifermento.
		
		Riguardo i componenti presenti in magazzino si vuole conoscere la quantità presente, il prezzo a cui sono stati acquistati, il periodo di validità (dopo il quale il componente diventa inutilizzabile per l’usura o per la caduta in obsolescenza). Inoltre sarà possibile stabilire, di ogni componente, anche la quantità minima che deve essere sempre presente in magazzino, in base all’analisi dell’uso. 
		
		Riguardo i preventivi si vogliono conoscere la data di emissione, il costo stimato, i tempi d’esecuzione stimati, la data in cui dovrebbe cominciare il lavoro, l’ammontare di un eventuale acconto versato. Se il cliente ha bisono di una riparazione, sarà utile avere a disposizione anche una piccola descrizione con una stima del danno e una stima del costo della manodopera. Se il cliente necessita dell’installazione di un nuovo impianto, sarà necessario tenere conto anche della tipologia di impianto richiesta (metano o GPL), tipologia del sistema di alimentazione (iniezione o aspirazione). Di ogni componente previsto per l’installazione è necessario sapere anche la loro ubicazione nell’autovettura (vano motore o vano bagagliaio). 
		
		Riguardo le prestazioni eseguite, a fronte di un preventivo, si vuole conoscere il preventivo di riferimento, i tempi effettivi di esecuzione, la data in cui è stato finito il lavoro, i componenti che sono stati effettivamente utilizzati, il costo di eventuali servizi aggiuntivi, il totale imponibile, i lavoratori che hanno le hanno eseguite. Si vuole tenere traccia anche dei pagamenti ricevuti per ogni transazione che, vista la possibilità di rateizzazione, saranno generalmente più di uno per prestazione. 
		
		Si vogliono conoscere anche i dati relativi alle transazioni monetarie, ovvero la quota, la data di emissione o ricevimento ed una eventuale causale ove necessario specificare qualche informazione aggiuntiva.
		
	\subsection{Glossario dei Termini}
	
		\marginpar{Aggiungere l'header della tabella a tutte le pagine}
						
		\begin{longtable}{| p{2.5cm} | p{4.5cm} | p{2cm} | p{2.5cm} |}
				
			\hline
				Termine & Descrizione & Sinonimi & Collegamenti \\ 
				\hline
				Cliente & 
				Persona fisica o giuridica che abbia avuto rapporti con l'azienda.
				&&\\ \hline
				Persona &
				Cliente non dotato di Partita IVA.
				&&\\ \hline
				Azienda &
				Cliente dotato di Partita IVA.
				&&\\ \hline
				Autovettura &
				Automobile di un cliente che debba subire o abbia già subito un intervento da parte dei lavoratori dell’azienda. &
				Automobile &
				Cliente, Preventivo
				\\ \hline
				Preventivo &
				Stima dei costi, dei tempi e dei componenti necessari relativi all’esecuzione di un intervento su un’autovettura. & &
				Autovettura, Componente, Prestazione
				\\ \hline
				Preventivo di Riparazione &
				Preventivo relativo alla riparazione di un guasto di un’autovettura & &
				Preventivo 
				\\ \hline
				Preventivo d'Installazione & 
				Preventivo relativo all’installazione di un nuovo impianto in un’autovettura. & &
				Preventivo
				\\ \hline
				Componente &
				Qualsiasi oggetto fisico necessario alla corretta esecuzione di una riparazione o di una installazione di un impianto su di un’autovettura &
				Prodotto & 
				Fornitore 
				\\ \hline
				Fornitore & 
				Azienda che abbia fornito all’officina qualsiasi tipo di componente necessario. 
				&&\\ \hline
				Dipendente & 
				Persona fisica che abbia lavorato per l’officina &
				Lavoratore, Operatore &
				\\ \hline
				Transazione &
				Flusso di denaro uscente o entrante nella cassa dell’attività. &
				Flusso di Cassa &
				\\ \hline
				Prestazione &
				Attività eseguita dai lavoratori dell’officina su di un’autovettura. & &
				Preventivo
				\\ \hline
				Pagamento &
				Transazione di denaro entrante a seguito di una prestazione fornita ad un cliente. &&
				Prestazione
				\\ \hline
				Collaudo &
				Attività relativa alla verifica specifica del corretto funzionamento del serbatoio installato con il nuovo impianto (che sia a metano o gpl). 
				&& \\ \hline
				Revisione & 
				Attività di verifica del corretto funzionamento dei tutte le parti dell’autovettura. 
				&& \\ \hline
				Retribuzione Oraria & 
				Ammontare della retribuzione di un dipendente per ogni ora di lavoro. && 
				Dipendente
				\\ \hline
				Modalità di Riscossione &
				Modalità, indicata dal dipendente, con cui quest’ultimo riceve lo stipendio. &&
				Dipendente 
				\\ \hline
				Impianto & 
				Infrastruttura di alimentazione di un’autovettura.
				&&\\ \hline
				Imponibile &
				Somma di denaro su cui vanno calcolate le imposte previste per legge. &&
				Pagamento 
				\\ \hline
				
		\end{longtable}
		
	\subsection{Eliminazione delle Ambiguità Presenti}
	\subsection{Strutturazione dei Requisiti}
	\subsection{Specifica delle Operazioni}