%!TEX root = Progetto.tex
\section{Codifica Sql e Testing} % (fold)
\label{sec:codifica_sql_e_testing}

Di seguito è riportata la definizione dello schema in linguaggio sql così come è implementato nel dump. Si allegano, per ogni tabella, degli screenshot dal terminale.

Il DBMS utilizzato (MySQL 5) nativamente non supporta la definizione di vincoli d'integrità personalizzabili. Per ovviare a questa limitazione, nell'implementazione completa dello schema (riportata nel dump) si è fatto un larghissimo uso di trigger che implementano la logica dei vincoli.

	\subsection{Definizione dello schema e screenshot successivi all'inserimento di dati}
		% Cliente

		\pagebreak
	\subsection{Codifica delle operazioni}
		Di seguito vengono riportate le implementazioni delle operazioni effettuabili sulla base di dati. La loro organizzazione non è lineare come l'elenco presentato in fase progettuale, ma rispecchia quelli che sono i reali casi d'uso.

		\begin{enumerate}

			\item Inserimento di un nuovo cliente. A tale inserimento corrisponde sempre l'inserimento e l'associazione di una nuova autovettura e di almeno un recapito.
			\begin{lstlisting}
/** Cliente non dotato di partita iva */
INSERT INTO Cliente (CF_PIVA, Nome, Cognome, Citta, 
	Via, Civico, CAP, NDocId) 
VALUES (...);

/** Cliente dotato di partita iva */
INSERT INTO Cliente (CF_PIVA, RagioneSociale, Citta, 
	Via, Civico, CAP) 
VALUES (...);

/** Inserimento dell'autovettura */
INSERT INTO Autovettura (Targa, Telaio, Marca, Modello, 
	Cilindrata, AnnoImmatricolazione, UltimoCollaudo, 
	UltimaRevisione, Cliente) 
VALUES (..., <CF_PIVA del cliente appena inserito>);

/** Inserimento di un recapito */
INSERT INTO Recapito (Recapito, Tipo)
VALUES (...);

/** Associazione di un recapito inserito al cliente */
INSERT INTO RubricaCliente (Recapito, Cliente)
VALUES (<Codice del recapito>, <CF_PIVA del cliente>);
				\end{lstlisting}

			L'aggiunta di un recapito ad un cliente è un'operazione molto frequente. Per questo è stata creata una procedura che si occupa dell'inserimento e dell'associazione dello stesso:
				\begin{lstlisting}
CREATE PROCEDURE add_recapito_cliente(
  IN cf_piva  VARCHAR(16),
     recapito VARCHAR(200),
     tipo     ENUM('telefono',
                   'fax',
                   'tel_fax',
                   'sito_web',
                   'email')
)
  BEGIN
    DECLARE EXIT HANDLER FOR SQLEXCEPTION
    BEGIN
      ROLLBACK;
      /* La procedura throw_error si occupa di lanciare 
       * un segnale d'errore con codice 45000
       * e con il messaggio specificato come argomento
       */
      CALL throw_error('Recapito già registrato');
    END;
    START TRANSACTION;
    INSERT INTO Recapito (Recapito, Tipo) VALUES (recapito, tipo);
    SELECT LAST_INSERT_ID()
    INTO @last_id;
    INSERT INTO RubricaCliente (Recapito, Cliente) VALUES (@last_id, cf_piva);
    COMMIT;
  END;;
				\end{lstlisting}

			\item Inserimento di un nuovo fornitore. All'inserimento di un nuovo fornitore corrisponde l'inserimento dei relativi recapiti.
			\begin{lstlisting}
/** Inserimento del fornitore */
INSERT INTO Fornitore (PIVA, RagioneSociale, TempiConsegna,
	ModPagamento, IBAN, Citta, Via, Civico, CAP) 
VALUES (...);

/** Associazione di un recapito inserito al fornitore
	(L'operazione di inserimento di un nuovo recapito
	è identica a quella presentata nel caso d'uso 
	precedente) */
INSERT INTO RubricaFornitore (Recapito, Fornitore)
VALUES (<Codice del recapito>, <PIVA del fornitore>);
			\end{lstlisting}

			Come nel caso precedente, anche per il fornitore esiste una procedura per l'aggiunta e l'associazione di un recapito.
			Gli inserimenti diventano molto più compatti:
			\begin{lstlisting}
CALL add_recapito_fornitor(<PIV del fornitore>,
	<recapito>, <tipo del recapito>);
			\end{lstlisting}

			\item Inserimento di un nuovo operatore. Anche in questo caso bisognerà aggiungere uno o più recapiti a quest'ultimo. Presentiamo inoltre anche l'istruzione per inserire un nuovo turno di lavoro.
			\begin{lstlisting}
/** Inserimento di un operatore con stipendio fisso */
INSERT INTO Operatore (CF, Nome, Cognome, Citta, Via, 
	Civico, CAP, DataNasc, ComuneNasc, ProvinciaNasc, 
	Stipendio, ModRiscossione, IBAN)
VALUES (...);

/** Inserimento di un operatore con retribuzione
	oraria */
INSERT INTO Operatore (CF, Nome, Cognome, Citta, Via,
	Civico, CAP, DataNasc, ComuneNasc, ProvinciaNasc, 
	RetribuzioneH, ModRiscossione, IBAN)
VALUES (...);

/** Inserimento di un nuovo turno di lavoro */
INSERT INTO Turno (Operatore, Data, OraInizio, OraFine)
VALUES (<CF dell'operatore>, ...);
			\end{lstlisting}

			\item Registrazione di un nuovo componente.
			\begin{lstlisting}
INSERT INTO Componente (Nome, QuantitaMin, Validita, 
	PrezzoVendita) 
VALUES (...);
			\end{lstlisting}

			\item Inserimento di un nuovo ordine. Questa operazione si articola in più parti: innanzi tutto bisogna creare una nuova istanza nella relazione \emph{Ordine}, quindi creare le istanze relative alle forniture dell'ordeine, aggiornare il magazzino ed inserire una nuova transazione a pagamento avvenuto dell'ordine.

			\begin{lstlisting}

			\end{lstlisting}

		\end{enumerate}