%!TEX root = Progetto.tex
\section{Codifica Sql e Testing} % (fold)
\label{sec:codifica_sql_e_testing}

Di seguito è riportata la definizione dello schema in linguaggio sql così come è implementato nel dump. Si allegano, per ogni tabella, degli screenshot dal terminale.

Il DBMS utilizzato (MySQL 5) nativamente non supporta la definizione di vincoli d'integrità personalizzabili. Per ovviare a questa limitazione, nell'implementazione completa dello schema (riportata nel dump) si è fatto un larghissimo uso di trigger che implementano la logica dei vincoli.

	\subsection{Definizione dello schema e screenshot successivi all'inserimento di dati}
		% Cliente
		\begin{lstlisting}
CREATE TABLE Cliente (
  CF_PIVA        VARCHAR(16) PRIMARY KEY,
  Nome           VARCHAR(80),
  Cognome        VARCHAR(80),
  RagioneSociale VARCHAR(80),
  Citta          VARCHAR(50) NOT NULL,
  Via            VARCHAR(50) NOT NULL,
  Civico         VARCHAR(10) NOT NULL,
  CAP            VARCHAR(5)  NOT NULL,
  NDocId         VARCHAR(9)
);
		\end{lstlisting}

		\begin{lstlisting}
CREATE TABLE Fornitore (
  PIVA           VARCHAR(11) PRIMARY KEY,
  RagioneSociale VARCHAR(80)                 NOT NULL,
  TempiConsegna  INTEGER,
  ModPagamento   ENUM('bonifico', 'assegno') NOT NULL,
  IBAN           VARCHAR(27),
  Citta          VARCHAR(50)                 NOT NULL,
  Via            VARCHAR(50)                 NOT NULL,
  Civico         VARCHAR(10)                 NOT NULL,
  CAP            VARCHAR(5)                  NOT NULL
);

CREATE TABLE Operatore (
  CF             VARCHAR(16) PRIMARY KEY,
  Nome           VARCHAR(80)                             NOT NULL,
  Cognome        VARCHAR(80)                             NOT NULL,
  Citta          VARCHAR(50)                             NOT NULL,
  Via            VARCHAR(50)                             NOT NULL,
  Civico         VARCHAR(10)                             NOT NULL,
  CAP            VARCHAR(5)                              NOT NULL,
  DataNasc       DATE                                    NOT NULL,
  ComuneNasc     VARCHAR(50)                             NOT NULL,
  ProvinciaNasc  VARCHAR(2)                              NOT NULL,
  Stipendio      DECIMAL(6, 2),
  RetribuzioneH  DECIMAL(4, 2),
  ModRiscossione ENUM('bonifico', 'assegno', 'contanti') NOT NULL,
  IBAN           VARCHAR(27)
);

CREATE TABLE Transazione (
  Codice INTEGER AUTO_INCREMENT PRIMARY KEY,
  Quota  DECIMAL(10, 2) NOT NULL,
  Data   DATE           NOT NULL
);

CREATE TABLE Autovettura (
  Targa                VARCHAR(8) PRIMARY KEY,
  Telaio               VARCHAR(17),
  Marca                VARCHAR(50)  NOT NULL,
  Modello              VARCHAR(100) NOT NULL,
  Cilindrata           INTEGER,
  AnnoImmatricolazione INTEGER      NOT NULL,
  UltimoCollaudo       DATE,
  UltimaRevisione      DATE,
  Cliente              VARCHAR(16)  NOT NULL,
  FOREIGN KEY (Cliente) REFERENCES Cliente (CF_PIVA)
);

CREATE TABLE Preventivo (
  Codice           INTEGER       AUTO_INCREMENT PRIMARY KEY,
  DataEmissione    DATE       NOT NULL,
  DataInizio       DATE       NOT NULL,
  Categoria        ENUM('riparazione',
                        'installazione_impianto_metano',
                        'installazione_impianto_gpl',
                        'collaudo',
                        'revisione'
  )                           NOT NULL,
  Sintomi          VARCHAR(300),
  SisAlimentazione ENUM('aspirazione', 'iniezione'),
  TempoStimato     INTEGER,
  CostoComponenti  DECIMAL(8, 2) DEFAULT 0,
  Manodopera       DECIMAL(7, 2) DEFAULT 0,
  ServAggiuntivi   DECIMAL(7, 2) DEFAULT 0,
  Autovettura      VARCHAR(8) NOT NULL,
  Acconto          INTEGER,
  FOREIGN KEY (Autovettura) REFERENCES Autovettura (Targa),
  FOREIGN KEY (Acconto) REFERENCES Transazione (Codice)
);

CREATE TABLE Componente (
  Codice        INTEGER AUTO_INCREMENT PRIMARY KEY,
  Nome          VARCHAR(150)  NOT NULL,
  QuantitaMin   INTEGER DEFAULT 0,
  Validita      INTEGER DEFAULT 0,
  PrezzoVendita DECIMAL(6, 2) NOT NULL
);

CREATE TABLE Previsione (
  Componente     INTEGER       NOT NULL,
  Preventivo     INTEGER       NOT NULL,
  Ubicazione     ENUM('motore', 'bagagliaio'),
  Quantita       INTEGER       NOT NULL,
  PrezzoUnitario DECIMAL(6, 2) NOT NULL,
  PRIMARY KEY (Componente, Preventivo),
  FOREIGN KEY (Componente) REFERENCES Componente (Codice),
  FOREIGN KEY (Preventivo) REFERENCES Preventivo (Codice)
);

CREATE TABLE Prestazione (
  Preventivo       INTEGER PRIMARY KEY,
  TempiEsecuzione  INTEGER NOT NULL,
  Procedimento     TEXT,
  DataFine         DATE    NOT NULL,
  Malfunzionamento VARCHAR(300),
  Manodopera       DECIMAL(7, 2) DEFAULT 0,
  ServAggiuntivi   DECIMAL(7, 2) DEFAULT 0,
  FOREIGN KEY (Preventivo) REFERENCES Preventivo (Codice)
);

CREATE TABLE Occupazione (
  Prestazione INTEGER     NOT NULL,
  Operatore   VARCHAR(16) NOT NULL,
  PRIMARY KEY (Prestazione, Operatore),
  FOREIGN KEY (Prestazione) REFERENCES Prestazione (Preventivo),
  FOREIGN KEY (Operatore) REFERENCES Operatore (CF)
);

CREATE TABLE Ordine (
  Codice        INTEGER       AUTO_INCREMENT PRIMARY KEY,
  DataEmissione DATE        NOT NULL,
  DataConsegna  DATE,
  Imponibile    DECIMAL(9, 2) DEFAULT 0,
  Fornitore     VARCHAR(11) NOT NULL,
  Versamento    INTEGER     NOT NULL,
  FOREIGN KEY (Fornitore) REFERENCES Fornitore (PIVA),
  FOREIGN KEY (Versamento) REFERENCES Transazione (Codice)
);

CREATE TABLE Fornitura (
  Codice         INTEGER AUTO_INCREMENT PRIMARY KEY,
  Quantita       INTEGER       NOT NULL,
  PrezzoUnitario DECIMAL(8, 2) NOT NULL,
  Componente     INTEGER       NOT NULL,
  Ordine         INTEGER       NOT NULL,
  FOREIGN KEY (Componente) REFERENCES Componente (Codice),
  FOREIGN KEY (Ordine) REFERENCES Ordine (Codice)
);

CREATE TABLE Utilizzo (
  Prestazione    INTEGER       NOT NULL,
  Fornitura      INTEGER       NOT NULL,
  PrezzoUnitario DECIMAL(8, 2) NOT NULL,
  Quantita       INTEGER       NOT NULL,
  PRIMARY KEY (Prestazione, Fornitura),
  FOREIGN KEY (Prestazione) REFERENCES Prestazione (Preventivo),
  FOREIGN KEY (Fornitura) REFERENCES Fornitura (Codice)
);

CREATE TABLE Magazzino (
  Componente INTEGER NOT NULL,
  Fornitura  INTEGER NOT NULL,
  Quantita   INTEGER NOT NULL,
  PRIMARY KEY (Componente, Fornitura),
  FOREIGN KEY (Componente) REFERENCES Componente (Codice),
  FOREIGN KEY (Fornitura) REFERENCES Fornitura (Codice)
);

CREATE TABLE Fattura (
  Numero        INTEGER                                 NOT NULL,
  Anno          INTEGER                                 NOT NULL,
  Imponibile    DECIMAL(10, 2)                          NOT NULL,
  Sconto        DECIMAL(4, 2) DEFAULT 0                 NOT NULL,
  Incentivi     DECIMAL(8, 2) DEFAULT 0                 NOT NULL,
  DataEmissione DATE                                    NOT NULL,
  DataScadenza  DATE                                    NOT NULL,
  TipoPag       ENUM('bonifico', 'assegno', 'contanti') NOT NULL,
  StatoPag      BOOLEAN DEFAULT FALSE                   NOT NULL,
  SisPag        ENUM('rimessa_diretta', 'rimessa_differita')
    DEFAULT 'rimessa_diretta'                           NOT NULL,
  Prestazione   INTEGER                                 NOT NULL,
  Transazione   INTEGER,
  PRIMARY KEY (Numero, Anno),
  FOREIGN KEY (Prestazione) REFERENCES Prestazione (Preventivo),
  FOREIGN KEY (Transazione) REFERENCES Transazione (Codice)
);

CREATE TABLE Turno (
  Operatore VARCHAR(16) NOT NULL,
  Data      DATE        NOT NULL,
  OraInizio TIME        NOT NULL,
  OraFine   TIME        NOT NULL,
  PRIMARY KEY (Operatore, Data, OraInizio),
  FOREIGN KEY (Operatore) REFERENCES Operatore (CF)
);

CREATE TABLE Stipendio (
  Transazione INTEGER PRIMARY KEY,
  Operatore   VARCHAR(16) NOT NULL,
  FOREIGN KEY (Transazione) REFERENCES Transazione (Codice),
  FOREIGN KEY (Operatore) REFERENCES Operatore (CF)
);

CREATE TABLE Recapito (
  Codice   INTEGER AUTO_INCREMENT PRIMARY KEY,
  Recapito VARCHAR(200)  NOT NULL,
  Tipo     ENUM('telefono',
                'fax',
                'tel_fax',
                'sito_web',
                'email') NOT NULL,
  UNIQUE (Recapito)
);

CREATE TABLE RubricaCliente (
  Recapito INTEGER     NOT NULL,
  Cliente  VARCHAR(16) NOT NULL,
  PRIMARY KEY (Recapito, Cliente),
  FOREIGN KEY (Cliente) REFERENCES Cliente (CF_PIVA),
  FOREIGN KEY (Recapito) REFERENCES Recapito (Codice)
);

CREATE TABLE RubricaFornitore (
  Recapito  INTEGER     NOT NULL,
  Fornitore VARCHAR(11) NOT NULL,
  PRIMARY KEY (Recapito, Fornitore),
  FOREIGN KEY (Fornitore) REFERENCES Fornitore (PIVA),
  FOREIGN KEY (Recapito) REFERENCES Recapito (Codice)
);

CREATE TABLE RubricaOperatore (
  Recapito  INTEGER     NOT NULL,
  Operatore VARCHAR(16) NOT NULL,
  PRIMARY KEY (Recapito, Operatore),
  FOREIGN KEY (Operatore) REFERENCES Operatore (CF),
  FOREIGN KEY (Recapito) REFERENCES Recapito (Codice)
);
		\end{lstlisting}


	\subsection{Codifica delle operazioni}
		Vengono riportate di seguito tutte le operazioni previste nelle fasi progettuali precedenti (sorvolando quelle banali).